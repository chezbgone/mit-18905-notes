\documentclass{standalone}
\usepackage{chez}

\begin{document}
\chapter{October 28, 2020}

\begin{question}
  What is the homology of a tensor product of chain complexes?
\end{question}

\begin{theorem}[K\"unneth]
  Suppose that \(C_*\) and \(D_*\) are chain complexes of \(\ZZ\)-modules
  and suppose \(C_*\) is a complex of free \(\ZZ\)-modules. Then
  \[
    H_n(C_* \otimes_\ZZ D_*)
      = \parens*{\bigoplus_{p + q = n} H_p(C_*) \otimes_\ZZ H_q(D_*)} \oplus
        \parens*{\bigoplus_{p + q = n-1} \Tor_1^{\ZZ}(H_p(C_*), H_q(D_*))}.
  \]
\end{theorem}

Note the similarity between this theorem and the universal coefficient theorem.
We have a guess for the tensor product in the first term,
and then correction terms described by the \(\Tor\) functor.
In fact, the universal coefficient theorem is a special case of this theorem,
where \(D_*\) is concentrated in degree \(0\).

\begin{proof*}{Proof idea 1}
  Note that \(C_* \otimes D_*\) calculates
  \(C_* \otimes^{\mathbb L} D_*\) in \(\der(\ZZ)\).
  Its homology will not change if we replace \(D_*\) with
  any chain complex isomorphic to \(D_*\) in \(\der(\ZZ)\).
  In particular, replace \(D_*\) with the chain complex \(D_*'\)
  such that \((D_*')_q = H_q(D_*)\) and all of the maps are \(0\) maps.
  Note that \(D_*'\) is a direct sum of
  chain complexes concentrated in single degrees,
  so we can reduce the theorem to the universal coefficient theorem.
\end{proof*}

However, we may want to give a proof that does not use
the derived category, since we have not talked about those rigorously.
This proof will still however, have the same spirit, since we will
reduce the proof to the universal coefficient theorem.
\begin{proof*}{Proof idea 2}
  Consider the new chain complex \(Z(D_*)\) where
  \[
    Z(D_*)_n = Z_n(D_*) = \ker(\partial \colon D_n \to D_{n-1})
  \]
  and all \(\partial\) maps in \(Z(D_*)\) are \(0\).
  Note that the is an inclusion of chain complexes \(Z(D_*) \to D_*\).
  \(Z(D_*)\) is a direct sum of chain complexes concentrated in single degrees.
  
  Note that the quotient chain complex \(D_*/Z(D_*)\) also has boundary maps
  that are the zero maps.
  Therefore, \(D_* / Z(D_*)\) is also a direct sum of chain complexes
  concentrated in single degrees.

  Consider the short exact sequence of chain complexes
  \[
    \begin{tikzcd}
      0 \arrow[r] &
      C_* \otimes_\ZZ Z(D_*) \arrow[r] &
      C_* \otimes_\ZZ D_* \arrow[r] &
      C_* \otimes_\ZZ D_*/Z(D_*) \arrow[r] &
      0
    \end{tikzcd}
  \]
  This gives the long exact sequence in homology from the snake lemma
  to write \(H_n(C_* \otimes_\ZZ D_*)\) in terms of
  \(H_n(C_* \otimes_\ZZ Z(D_*))\) and \(H_{n-1}(C_* \otimes_\ZZ D_*/Z(D_*))\).
  This reduces to the universal coefficient theorem.
\end{proof*}

\begin{remark}
  This extends to all PIDs, not just \(\ZZ\).
  Suppose that \(R\) is a PID and
  \(C_*\) and \(D_*\) are chain complexes of \(R\)-modules
  such that \(C_*\) is a chain complex of free \(R\)-modules.
  Then
  \[
    H_n(C_* \otimes_R D_*) \iso
      \parens*{\bigoplus_{p + q = n} H_p(C_*) \otimes_R H_q(D_*)} \oplus
      \parens*{\bigoplus_{p + q = n-1} \Tor_1^R(H_p(C_*), H_q(D_*))}.
  \]
\end{remark}
The hypothesis that \(R\) has to be a PID comes from how we needed a
\(2\)-term free resolution of \(M\) in the proof of
the universal coefficient theorem.

In particular, if \(R\) is a field, then
\[
  H_n(C_* \otimes_R D_*) \iso \bigoplus_{p+q = n} H_p(C_*) \otimes_R H_q(D_*).
\]
This is because all free resolutions over a field are not interesting,
so all of the \(\Tor\) functors vanish.

\begin{theorem}[Eilenberg-Zilber]<eilenberg-zilber>
  Suppose \(X\) and \(Y\) are topological spaces
  and \(R\) is a commutative ring.
  Then \(S_*(X \times Y; R)\) is quasi-isomorphic to
  \(S_*(X; R) \otimes_R S_*(Y; R)\).
\end{theorem}
\begin{proof*}{Proof idea 1}
  If \(X\) and \(Y\) are CW complexes,
  then we can examine the CW structure on \(X \times Y\).
  Rather than developing the theory of product CW structures,
  we will directly prove the theorem instead,
  since not all topological spaces have a natural CW structure.
\end{proof*}

Before we prove the theorem, let's do some examples.
\begin{example}
  Let's compute the groups \(H_q(\RP^2 \times \RP^2; \FF_2)\).
  Recall that we have
  \[
    H_q(\RP^2; \FF_2) = \begin{cases*}
      \FF_2 & \(q = 0, 1, 2\) \\[-1ex]
      0 & otherwise.
    \end{cases*}
  \]
  Since \(\FF_2\) is a field, this means
  \begin{align*}
    H_0(\RP^2 \times \RP^2; \FF_2)
      &\iso \bigoplus_{p + q = 0} H_p(\RP^2; \FF_2) \otimes_{\FF_2}
                                  H_q(\RP^2; \FF_2) \\
      &\iso H_0(\RP^2; \FF_2) \otimes_{\FF_2} H_0(\RP^2; \FF_2) \\
      &\iso \FF_2 \otimes_{\FF_2} \FF_2 \\
      &\iso \FF_2 \\
    %
    H_1(\RP^2 \times \RP^2; \FF_2)
      &\iso \bigoplus_{p + q = 1} H_p(\RP^2; \FF_2) \otimes_{\FF_2}
                                  H_q(\RP^2; \FF_2) \\
      &\iso H_0(\RP^2; \FF_2) \otimes_{\FF_2} H_1(\RP^2; \FF_2) \oplus
            H_1(\RP^2; \FF_2) \otimes_{\FF_2} H_0(\RP^2; \FF_2) \\
      &\iso \FF_2 \otimes_{\FF_2} \FF_2 \oplus \FF_2 \otimes_{\FF_2} \FF_2 \\
    %
    H_2(\RP^2 \times \RP^2; \FF_2)
      &\iso (H_0 \otimes H_2) \oplus
            (H_1 \otimes H_1) \oplus
            (H_2 \otimes H_0) \\
      &\iso \FF_2 \oplus \FF_2 \oplus \FF_2 \\
    %
    H_3(\RP^2 \times \RP^2; \FF_2)
      &\iso (H_0 \otimes H_3) \oplus
            (H_1 \otimes H_2) \oplus
            (H_2 \otimes H_1) \oplus
            (H_3 \otimes H_0) \\
      &\iso (\FF_2 \otimes 0    ) \oplus
            (\FF_2 \otimes \FF_2) \oplus
            (\FF_2 \otimes \FF_2) \oplus
            (0     \otimes \FF_2) \\
      &\iso \FF_2 \oplus \FF_2 \\
    %
    H_4(\RP^2 \times \RP^2; \FF_4)
      &\iso (H_0 \otimes H_4) \oplus
            (H_1 \otimes H_3) \oplus
            (H_2 \otimes H_2) \oplus
            (H_3 \otimes H_1) \oplus
            (H_4 \otimes H_0) \\
      &\iso (\FF_2 \otimes 0    ) \oplus
            (\FF_2 \otimes 0    ) \oplus
            (\FF_2 \otimes \FF_2) \oplus
            (0     \otimes \FF_2) \oplus
            (0     \otimes \FF_2) \\
      &\iso \FF_2.
  \end{align*}
  This gives
  \[
    H_q(\RP^2 \times \RP^2; \FF_2) \iso \begin{cases}
      \FF_2                           & q = 0            \\[-1ex]
      \FF_2 \oplus \FF_2              & q = 1            \\[-1ex]
      \FF_2 \oplus \FF_2 \oplus \FF_2 & q = 2            \\[-1ex]
      \FF_2 \oplus \FF_2              & q = 3            \\[-1ex]
      \FF_2                           & q = 4            \\[-1ex]
      0                               & \text{otherwise.}
    \end{cases}
  \]
  One thing to note is that these groups are symmetric around \(q = 2\).
  This is called Poincar\'e duality, and we will prove it later.
\end{example}

\begin{remark}
  There is a diagonal map
  \begin{align*}
    \Delta \colon \RP^2 &\to \RP^2 \times \RP^2 \\[-1ex]
      x &\mapsto (x, x)
  \end{align*}
  where
  \begin{align*}
    H_2(\Delta) \colon H_2(\RP^2; \FF_2) &\to
                        H_2(\RP^2 \times \RP^2; \FF_2) \\[-1ex]
      \FF_2 &\to \FF_2 \oplus \FF_2 \oplus \FF_2.
  \end{align*}
  We would like the figure out how to describe this map.
  We will do this later in the course.
\end{remark}

To prove Eilenberg-Zilber (\cref{thm:eilenberg-zilber}),
we want to show \(S_*(X \times Y; R)\) is quasi-isomorphic to
\(S_*(X; R) \times_R S_*(Y, R)\).
We will show that the following map is the quasi-isomorphism.

\begin{definition}
  The \vocab{Alexander-Whitney map}
  \[
    A \colon S_*(X \times Y; R) \to S_*(X; R) \times S_*(Y; R)
  \]
  is a chain map defined as follows:
  For each integer \(n\) and \(p + q = n\), let
  \[
    A \colon S_n(X \times Y; R) \to S_p(X; R) \otimes S_q(Y; R).
  \]
  It suffices to define \(A(\sigma)\) for \(\sigma \in \Sing_n(X \times Y)\).
  Note that \(\sigma \colon \Delta^n \to X \times Y\) is determined
  by two maps \(\alpha \colon \Delta^n \to X\)
  and \(\beta \colon \Delta^n \to Y\).
  Let
  \[
    A(\sigma) = (\alpha {\restriction}_{\Delta^p}) \otimes
                (\beta {\restriction}_{\Delta^q})
  \]
  where the restriction to \(\Delta^p\) is given by the map
  \begin{align*}
    \Delta^p &\to \Delta^n \\[-1ex]
       [e_0: e_1: \dots: e_p] &\mapsto [e_0: e_1: \dots: e_p: 0: \dots: 0]
  \end{align*}
  and the restriction to \(\Delta^q\) is given by the map
  \begin{align*}
    \Delta^q &\to \Delta^n \\[-1ex]
       [e_0: e_1: \dots: e_q] &\mapsto [0: \dots: 0: e_0: e_1: \dots: e_q].
  \end{align*}
\end{definition}








\end{document}
