\documentclass{standalone}
\usepackage{chez}

\begin{document}
\chapter{October 26, 2020}

\subsection{Homology with coefficients}
Recall that if \(X\) is a topological space,
we have chain complex of \(R\)-modules \(S_*(X; R)\) with
the \(n\)th group being the free \(R\)-module generated by \(\Sing_n(X)\).
The \(\partial\) maps are given by the alternating sums of the \(d_i\) maps.

\begin{note}
  \(S_*(X; R)\) may be alternatively described as \(S_*(X) \otimes_\ZZ R\),
  which also calculates the derived tensor product.
\end{note}

We can even make a more general definition
\begin{definition}
  Suppose \(M\) is an abelian group and \(X\) is a topological space.
  We can define \(S_*(X; M) \coloneqq S_*(X) \otimes_\ZZ M\)
  and \(H_q(X; M)\) to be \(H_q\) of \(S_*(X; M)\).
\end{definition}

If \(R\) is a commutative ring and \(M\) is an \(R\)-module,
then \(H_q(X; M)\) acquires the structure of an \(R\)-module.

\subsection{Cohomology}
Suppose \(X\) is a topological space and \(M\) is an abelian group.
We can make a chain complex, concentrated in non-positive degrees,
with the \((-n)\)th term
\(
  \ul\Hom_{\cat{\ZZ\text{-}mod}}(S_n(X), M) % chktex 35
\).
This new chain complex calculates \(\ul\Hom_{\der(\ZZ)}(S_*(X), M)\).

\begin{definition}
  The \((-q)\)th homology group of the above chain complex is denoted
  \(H^q(X; M)\), and is called the \(q\)th \vocab{cohomology group}
  of \(X\) with coefficients in \(M\).
\end{definition}

Both homology coefficients and cohomology with coefficients can
be determined by homology with integer coefficients.
However, these may be easier to compute, so they can be useful.

\subsection{Universal coefficient theorems}
\begin{theorem}[For cohomology]
  Let \(X\) be a topological space, and \(M\) be an abelian group.
  For any integer \(q\), there is an isomorphism
  \[
    H^q(X; M) \iso \ul\Hom_\cAb(H_q(X), M) \oplus \Ext^1_\ZZ(H_{q-1}(X), M).
  \]
\end{theorem}
Note that on the right side, we have expressions that only deal
with homology with integer coefficients, and on the left
we have homology with coefficients in an arbitrary abelian group \(M\).

\begin{theorem}[For homology]<homology-universal-coefficient>
  Let \(C_*\) be a chain complex of free \(\ZZ\)-modules
  (e.g.\ \(C_* = S_*(X)\) or \(C_* = \Ccell_*(X)\))
  and \(M\) is an abelian group.
  Then there is a natural short exact sequence
  \[
    \begin{tikzcd}
      0 \arrow[r] &
      H_q(C_*) \otimes_\ZZ M \arrow[r] &
      H_q(C_* \otimes_\ZZ M) \arrow[r] &
      \Tor_1(H_{q-1}(C_*), M) \arrow[r] &
      0
    \end{tikzcd}
  \]
  and furthermore an isomorphism
  \[
    H_q(C_* \otimes_\ZZ M)
      \iso H_q(C_*) \otimes_\ZZ M \oplus \Tor_1^{\ZZ}(H_{q-1}(C_*), M).
  \]
\end{theorem}
Note that similarly to the previous theorem,
the left is calculating homology with coefficients
but the right side is about homology with integer coefficients.

\begin{adhoctheorem}{Warning}<warning:unnatural-isomorphism>
  This isomorphism in \cref{thm:homology-universal-coefficient} is not natural!
  While the short exact sequence is natural,
  the direct sum decomposition of the middle term is not natural.
\end{adhoctheorem}

Let's do some computations before we prove this.
\begin{example}
  Consider \(\RP^2\).
  Recall that
  \[
    \Ccell_*(\RP^2) = \Ccell_*(\RP^2; \ZZ) \iso \begin{tikzcd}
        \ZZ \ar[r, "2"] & \ZZ \ar[r, "0"] & \ZZ
    \end{tikzcd}
  \]
  and we have
  \[
    H_q(\RP^2) = \begin{cases*}
      \ZZ & \(q = 0\) \\[-1ex]
      \ZZ/2\ZZ{} & \(q = 1\) \\[-1ex]
      0 & otherwise.
    \end{cases*}
  \]
  We can then as what \(H_q(\RP^2; \FF_2)\) are.

  One way we can compute this is to calculate
  \(\Ccell_*(\RP^2) \otimes_\ZZ \FF_2\) directly.
  \[
    \begin{tikzcd}
      \FF_2 \ar[r, "2"] & \FF_2 \ar[r, "0"] & \FF_2
    \end{tikzcd}
  \]
  This gives
  \[
    H_q(\RP^2; \FF_2) = \begin{cases*}
      \FF_2 & \(q = 0, 1, 2\) \\[-1ex]
      0 & otherwise.
    \end{cases*}
  \]
  
  The other way we can compute this is to use
  the universal coefficients theorem.
  In particular, we have
  \begin{align*}
    H_2(\RP^2; \FF_2) &\iso H_2(\RP^2) \otimes_\ZZ \FF_2
                          \oplus_\ZZ \Tor_1(H_1(\RP^2), \FF_2) \\
      &\iso 0 \otimes_\ZZ F_2 \oplus \Tor_1(\FF_2, \FF_2) \\
      &\iso 0 \oplus H_1(\FF_2 \otimes^{\mathbb L}_\ZZ \FF_2).
  \end{align*}
  We know
  \begin{align*}
    H_1(\FF_2 \otimes^{\mathbb L}_\ZZ \FF_2)
      &= H_1\parens[\Big]{
        \parens[\Big]{\begin{tikzcd}[ampersand replacement=\&]
          \ZZ \ar[r, "2"] \& \ZZ
        \end{tikzcd}}
        \otimes_\ZZ \FF_2
      } \\
      &= H_1\parens[\Big]{
        \begin{tikzcd}[ampersand replacement=\&]
          \FF_2 \ar[r, "2"] \& \FF_2
        \end{tikzcd}} \\
      &= \FF_2,
  \end{align*}
  so \(H_2(\RP^2; \FF_2) = \FF_2\), as expected.

  Similarly, we have
  \begin{align*}
    H_1(\RP^2; \FF_2) &\iso H_1(\RP^2) \otimes_\ZZ \FF_2
                          \oplus_\ZZ \Tor_1(H_0(\RP^2), \FF_2) \\
      &\iso \FF_2 \otimes_\ZZ F_2 \oplus \Tor_1(\ZZ, \FF_2) \\
      &\iso \FF_2 \oplus H_1(\ZZ \otimes^{\mathbb L} \FF_2) \\
      &\iso \FF_2 \oplus H_1\parens[\Big]{
        \begin{tikzcd}[ampersand replacement=\&, column sep=scriptsize]
          \cdots \ar[r] \& 0 \ar[r] \& \FF_2 \ar[r] \& 0 \ar[r] \& \cdots
        \end{tikzcd}
      } \\
      &\iso \FF_2 \oplus 0 = \FF_2,
  \end{align*}
  as desired.
\end{example}

\begin{example}
  Let's compute \(H_2(S^2; \FF_3)\). We have
  \begin{align*}
    H_2(S^2; \FF_3) &\iso H_2(S^2) \otimes_\ZZ \FF_3
                          \oplus_\ZZ \Tor_1(H_1(S^2), \FF_3) \\
      &\iso \ZZ \otimes_\ZZ F_3 \oplus \Tor_1(0, \FF_3) \\
      &\iso \FF_3 \oplus H_1(0 \otimes^{\mathbb L} \FF_2) \\
      &\iso \FF_2 \oplus H_1\parens[\big]{
        \begin{tikzcd}[ampersand replacement=\&, column sep=scriptsize]
          \cdots \ar[r] \& 0 \ar[r] \& 0 \ar[r] \& \cdots
        \end{tikzcd}
      } \\
      &\iso \FF_3 \oplus 0 = \FF_3,
  \end{align*}
\end{example}

\begin{proof}[\cref{thm:homology-universal-coefficient}]
  Suppose \(C_*\) is a chain complex of free \(\ZZ\)-modules and
          \(M\) is an abelian group.
  We want to calculate the homology groups of \(C_* \otimes_\ZZ M\).
  Consider a free resolution
  \[
    \begin{tikzcd}
      \cdots \ar[r] &
        0 \ar[r] &
        F_1 \ar[r] &
        F_0 \ar[r] &
        0 \ar[r] &
        \cdots
    \end{tikzcd}
  \]
  of \(M\).
  Since \(\ZZ\) is a PID, we can find this two-term free resolution.
  This gives us a short exact sequence of chain complexes
  \[
    \begin{tikzcd}
      0 \arrow[r] &
      C_* \otimes_\ZZ F_1 \arrow[r] &
      C_* \otimes_\ZZ F_0 \arrow[r] &
      C_* \otimes_\ZZ M \arrow[r] &
      0
    \end{tikzcd}
  \]
  which gives us a long exact sequence in homology
  \[
    \begin{tikzcd}[row sep=scriptsize]
      \cdots \ar[r] &
        H_q    (C_* \otimes F_1) \ar[d, symbol=\iso] \ar[r] &
        H_q    (C_* \otimes F_0) \ar[d, symbol=\iso] \ar[r] &
        H_q    (C_* \otimes M  ) \ar[d, symbol=\iso] \ar[r, "\partial"] &
        H_{q-1}(C_* \otimes F_1) \ar[d, symbol=\iso] \ar[r] &
        H_{q-1}(C_* \otimes F_0) \ar[d, symbol=\iso] \ar[r] &
        \cdots \\
      \cdots \ar[r] &
        H_q    (C_*) \otimes F_1 \ar[r] &
        H_q    (C_*) \otimes F_0 \ar[r] &
        H_q    (C_* \otimes M)   \ar[r] &
        H_{q-1}(C_*) \otimes F_1 \ar[r] &
        H_{q-1}(C_*) \otimes F_0 \ar[r] &
        \cdots
    \end{tikzcd}
  \]
  This long exact sequence implies the short exact sequence
  \[
    \begin{tikzcd}
      0 \arrow[r] &
        H_q(C_*) \otimes F_0/F_1 \arrow[r] &
        H_q(C_* \otimes M) \arrow[r] &
        \ker\brackets{
          H_{q-1}(C_*) \otimes F_1 \to H_{q-1}(C_*) \otimes F_0
        } \arrow[r] &
        0
    \end{tikzcd}
  \]
  Since \(F_1 \to F_0 \to M\) is a free resolution, we have
  \(F_0/F_1 \iso M\).
  Also, the third term of the short exact sequence is
  \(\Tor(H_{q-1}(C_*), M)\).
  This gives the short exact sequence
  \[
    \begin{tikzcd}
      0 \arrow[r] &
        H_q(C_*) \otimes M \arrow[r] &
        H_q(C_* \otimes M) \arrow[r] &
        \Tor_1(H_{q-1}(C_*), M) \arrow[r] &
        0
    \end{tikzcd}\pog
  \]
\end{proof}











\end{document}
