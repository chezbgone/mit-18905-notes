\documentclass{standalone}
\usepackage{chez}

\begin{document}
\chapter{November 04, 2020}

Let \(X\) be a topological space and
    \(R\) be a commutative ring.

We will discuss the fact that
\[
  H^*(X; R) = \bigoplus_{q \geq 0} H^q(X; R)
\]
is a \vocab{graded commutative ring}.

The fact that it is a \vocab{graded} abelian group means that
\begin{itemize}[nosep]
  \item Any class \(x \in H^q(X; R) \subseteq H^*(X; R)\) is said to be
  \vocab{homogeneous of degree \(q\)}.
  \item Every class \(x \in H^*(X; R)\) is a sum of
    finitely many homogeneous elements.
\end{itemize}

\begin{definition}
  The \vocab{cohomology cross product} is given by the following:
  If \(X\) and \(Y\) are topological spaces and \(R\) is a ring,
  there is a natural sequence of maps
  \[
    \begin{tikzcd}
    	H^*(X; R) \otimes_R H^*(Y; R) \ar[r, "f_1"] &
    		H^*(S^*(X; R) \otimes_R S^*(Y; R)) \ar[r, "f_2"] &
    		H^*(\ul\Hom_{\der(\ZZ)}(S_*(X) \otimes_\ZZ S_*(Y), R)) \ar[r, "f_3"] &
    		H^*(X \times Y; R)
    \end{tikzcd}
  \]
  which defines the cross product
  \[
    (\times) \colon H^*(X; R) \otimes_R H^*(Y; R) \to H^*(X \times Y; R).
  \]
\end{definition}
The cross product is an isomorphism in many cases, but not always.
In particular, it is an isomorphism if
  \(H_q(X; R)\) is a finitely generated free \(R\)-module
  for each \(q \in \ZZ\), or
  if \(H_q(Y; R)\) is a finitely generated free \(R\)-module
  for each \(q \in \ZZ\).
There are two separate assumptions going on.
The freeness condition is equivalent to \(f_1\) being an isomorphism,
which is related to the K\"unneth theorem,
and the finite generation is equivalent to \(f_2\) being an isomorphism.

\begin{proposition}
  We also have \(f_3\) is always an isomorphism.
\end{proposition}
\begin{proof}
  We have the definition
  \[
    H^*(X \times Y; R) = H^*(\ul\Hom_{\der(\ZZ)}(S_*(X \times Y), R)).
  \]
  Then by Eilenberg-Zilber,
  \[
    S_*(X \times Y) \iso S_*(X) \otimes_\ZZ S_*(Y),
  \]
  giving the result.
\end{proof}

Note that \(f_2\) is induced from a chain map
\[
  \ul\Hom_{\der(\ZZ)}(S_*(X), R) \otimes_\ZZ \ul\Hom_{\der(\ZZ)}(S_*(Y), R)
    \to \ul\Hom_{\der(\ZZ)}(S_*(X) \otimes_\ZZ S_*(Y), R),
\]
because by definition
\[
  \ul\Hom_{\der(\ZZ)}(S_*(X), R) \otimes_\ZZ \ul\Hom_{\der(\ZZ)}(S_*(Y), R)
    = S^*(X; R) \otimes_R S^*(Y; R).
\]
In particular,
\[
  f \otimes g \mapsto \begin{cases*}
    x \otimes y \mapsto (-1)^{pq} f(x) g(y) &
      \(\deg x = \deg f = p\)\: and\: \(\deg y = \deg g = q\) \\[-1ex]
    0 & otherwise.
  \end{cases*}
\]

If \(R\) is a PID, then \(f_1\) is the map from the K\"unneth theorem
\[
  \begin{tikzcd}
    0 \arrow[r] &
    H_*(C_*) \otimes_R H_*(D_*) \arrow[r, "f_1"] &
    H_*(C_* \otimes_R D_*) \arrow[r] &
    \text{\(\Tor\) terms} \arrow[r] &
    0
  \end{tikzcd}
\]
If \(R\) is not a PID, \(f_1\) still exists and is still natural,
but it is just not a part of an exact sequence.


\subsection{Remark about \texorpdfstring{\(H^0(X; R)\)}{H0(X; R)}} % chktex 13
If \(X\) is a topological space,
recall that \(\pi_0 X\) is the set of path components of \(X\),
and \(H_0(X; R)\) is the free \(R\)-module generated by \(\pi_0 X\).

The cohomology \(H^0(X; R)\) is the set \(\Hom_\cSet(\pi_0 X, R)\)
equipped with the natural \(R\)-module structure.
If \(\pi_0 X\) is finite, then
\(\Hom_\cSet(\pi_0 X, R) \iso \bigoplus_{\pi_0 X} R\).
However, if \(\pi_0 X\) is infinite, it is not a direct sum anymore,
because not every object is not necessarily a finite sum.


\subsection{The cohomology ring}
Recall that if \(X \to Y\) is a map in \(\cTop\), then there is a map
\[
  S_*(X) \to S_*(Y)
\]
and hence a map
\[
  \ul\Hom_{\der(\ZZ)}(S_*(Y), R) \to \ul\Hom_{\der(\ZZ)}(S_*(X), R),
\]
and hence a map
\[
  H^*(Y; R) \to H^*(X; R).
\]
Applying this to the diagonal map \(\Delta \colon X \to X \times X\),
this gives us a map
\[
  H^*(X \times X; R) \to H^*(X; R).
\]
If we compose this with a direct product, we get the map
\[
  H^*(X; R) \times H^*(X; R) \xrightarrow{\times}
  H^*(X \times X; R) \xrightarrow{H^*(\Delta; R)} H^*(X; R).
\]
This is called the \vocab{cup product} in cohomology
with coefficients in \(R\).

\begin{theorem}<cohomology-ring>
  The cup product makes \(H^*(X; R)\) into an graded-commutative ring.
  In particular:
  \begin{enumerate}[nosep]
    \item There is a unit \(1 \in H^0(X; R)\) such that
      if \(x \in H^*(X; R)\), \(1x = x1 = x\).
    \item If \(x, y, z \in H^*(X; R)\), then \((xy)z = x(yz)\)
      so we can talk about \(xyz\) unambiguously.
    \item (Graded-commutativity)
      If \(x \in H^p(X; R)\) and \(y \in H^q(X; R)\), then
      \(xy = (-1)^{pq} yx \in H^{p+q}(X; R)\).
  \end{enumerate}
\end{theorem}

Therefore, we have (modulo the sign issue),
a familiar object, namely a ring.
It is natural to ask what the ring properties tell us
about the space geometrically.


\begin{example}
  If we consider the torus \(T^2\) and the cohomology \(H^*(T^2; \FF_2)\),
  we get
  \[
    H^0 = \FF_2, \qquad
    H^1 = \FF_2 \otimes \FF_2, \qquad
    H^2 = \FF_2, \qquad
    H^3 = 0, \ldots.
  \]
  Note that these are the same as the homology groups,
  which is true by Poincar\'e duality.
  Let \(a\) and \(b\) be two generators of \(H_1 \iso H^1\).
  Then, we claim that \(ab \in H^2(T^2; \FF_2) = \FF_2\)
  is the element \(1\) because \(a\) and \(b\) intersect.
  In particular, even if we deform the generators,
  they intersect an odd number of times.

  In general, the cup product is saying something about how things intersect.
\end{example}










\end{document}
