\documentclass{standalone}
\usepackage{chez}

\begin{document}
\chapter{September 18, 2020}

\begin{definition}
  If \(A \subseteq X\) is a subspace, let \(H_n(X, A)\) be
  the \(n\)th homology of the quotient chain complex \(S_*(X)/S_*(A)\).
\end{definition}

The general strategy for computing \(H_m(X)\) is roughly the following:
\begin{enumerate}[nosep]
  \item Find a slightly simpler space \(A \subseteq X\).
  \item Relate \(H_m(A)\), \(H_m(X)\), and \(H_m(X, A)\).
  \item Relate \(H_m(X, A)\) to \(X_m(X/A)\).
\end{enumerate}

Before we make steps 2 and 3 precise, it is useful to introduce a new category.
\begin{definition}
  There is a category \(\cTop_2\) with
  \begin{gather*}
    \Obj \cTop_2 = \set*{\parbox{6cm}{%
      pairs \((X, A)\) where \(X\) is a topological space
      and \(A \subseteq X\) is a subspace
    }} \\
    \Hom_{\cTop_2}((X, A), (Y, B)) = \set*{\text{%
        continuous maps \(f \colon X \to Y\) where \(f(A) \subseteq B\)%
    }}
  \end{gather*}
\end{definition}

There is, for each \(m \geq 0\) a functor
\[
  H_m \colon \cTop_2 \to \cAb.
\]
There is also a functor
\[
  \cTop \to \cTop_2
\]
sending \(X\) to the pair \((X, \nullset)\). This lets us view \(\cTop\) as a
subcategory of \(\cTop_2\).

\section{Some algebra}
\begin{definition}
  Suppose
  \(
    \begin{tikzcd}[cramped, sep=small]
    	A \ar[r, "f"] &
    	B \ar[r, "g"] &
    	C
    \end{tikzcd}
  \)
  is a sequence of abelian groups with maps between them. We say this sequence
  is \vocab{exact} at \(B\) if \(\ker g = \img f\).
\end{definition}

\begin{example}
  \begin{itemize}[nosep]
    \item The sequence
      \(
        \begin{tikzcd}[cramped, sep=small]
          0 \ar[r] &
          A \ar[r, "f"] &
          B
        \end{tikzcd}
      \)
      is exact if and only if \(f\) is injective.
    \item The sequence
      \(
        \begin{tikzcd}[cramped, sep=small]
          A \ar[r, "f"] &
          B \ar[r] &
          0
        \end{tikzcd}
      \)
      is exact if and only if \(f\) is surjective.
  \end{itemize}
\end{example}

\begin{definition}
  A longer sequence is said to be \vocab{exact} if it is exact at every
  three-term subsequence.
\end{definition}

\begin{example}
  The sequence
  \[
    \begin{tikzcd}
    	0 \ar[r] &
    	\ZZ \ar[r, "2"] &
    	\ZZ \ar[r, "1"] &
    	\ZZ/2\ZZ \ar[r] &
    	0
    \end{tikzcd}
  \]
  is exact. Indeed, at the first \(\ZZ\), multiplication by \(2\) is injective;
  at the second \(\ZZ\), the kernel of the projection \(\ZZ \to \ZZ/2\ZZ\) is
  the same as the image of the multiplication by \(2\) map; and at
  the \(\ZZ/2\ZZ\), the projection \(\ZZ \to \ZZ/2\ZZ\) is surjective.

  In general, an exact sequence of the form
  \[
    \begin{tikzcd}
    	0 \ar[r] &
    	A \ar[r] &
    	B \ar[r] &
    	C \ar[r] &
    	0
    \end{tikzcd}
  \]
  is called a \vocab{short exact sequence}.
\end{example}

\begin{example}
  \[
    S_*(\nullset) = \parens*{
      \begin{tikzcd}[ampersand replacement=\&]
        \cdots \ar[r] \&
        0 \ar[r] \&
        0 \ar[r] \&
        0 \ar[r] \&
        0 \ar[r] \&
        \cdots
      \end{tikzcd}
    }
  \]
  is exact.

  \(S_*(*)\) is not exact. However, it is away from \(S_0(*)\).
\end{example}

\begin{theorem}[Five lemma]
  Suppose
  \[
    \begin{tikzcd}
    	A_4 \ar[r, "d"] \ar[d, "f_4"] &
        A_3 \ar[r, "d"] \ar[d, "f_3"] &
        A_2 \ar[r, "d"] \ar[d, "f_2"] &
        A_1 \ar[r, "d"] \ar[d, "f_1"] &
        A_0             \ar[d, "f_0"] \\
    	B_4   \ar[r, "d", swap] &
        B_3 \ar[r, "d", swap] &
        B_2 \ar[r, "d", swap] &
        B_1 \ar[r, "d", swap] &
        B_0
    \end{tikzcd}
  \]
  is a commutative diagram in \(\cAb\). Suppose that both rows are exact
  and that \(f_0, f_1, f_3, f_4\) are isomorphisms.
  Then \(f_2\) is also an isomorphism.
\end{theorem}
\begin{proof}
  We will show that \(f_2\) is surjective. The proof method we will use
  is called \vocab{diagram chasing}, which typically involves pointing at
  a diagram. To simulate the experience in text, we will use bullet points,
  as we often make seemingly unrelated claims that follow one after another.

  We will show surjectivity of \(f_2\) first.
  \begin{itemize}[nosep]
    \item Let \(b_2 \in B_2\).
    \item Since \(f_1\) is surjective, there exists \(a_1 \in A_1\) such that
    \(f_1 a_1 = d b_2\).
    \item By commutativity of the diagram, we have \(f_0 d a_1 = d f_1 a_2\).
    \item By exactness, \(d^2 = 0\), so \(f_0 d a_1 = d f_1 a_1 = ddb_2 = 0\).
    \item Since \(f_0\) is an isomorphism, \(d a_1 = 0\).
    \item By exactness, \(a_1 \in \ker d = \img d\)
    (where the \(d\)s have different indices),
    so there exists \(a_2 \in A_2\) such that \(d a_2 = a_1\).
    \item By commutativity, \(d f_2 a_2 = f_1 d a_2 = f_1 a_1 = d b_2\).
    \item Since \(d\) is a homomorphism, \(d(b_2 - f_2 a_2) = 0\).
    \item By exactness, there exists some \(b_3 \in B_3\) such that
    \(d b_3 = b_2 - f_2 a_2\).
    \item By commutativity, \(d f_3 = f_2 d\), so
    \(f_2 d a_3 = b_2 - f_2 a_2\).
    \item Rearranging, \(b_2 = f_2(a_2 + da_3)\), so \(f_2\) is surjective.
  \end{itemize}

  Now we will show that \(f_2\) is injective, i.e.\ \(\ker f_2 = 0\).
  \begin{itemize}[nosep]
    \item Let \(a_2 \in A_2\) such that \(f_2 a_2 = 0\), and therefore
    \(d f_2 a_2 = 0\). We want to show \(a_2 = 0\).
    \item By commutativity, we have \(f_1 d a_2 = 0\).
    \item Since \(f_1\) is injective, \(d a_2 = 0\).
    \item By exactness, there exists \(a_3 \in A_3\) such that \(d a_3 = a_2\).
    \item By commutativity, \(d f_3 a_3 = f_2 d a_3 = f_2 a_2 = 0\).
    \item By exactness, \(f_3 a_3 \in \ker d\), so there exists \(b_4 \in B_3\)
    such that \(d b_4 = f_3 a_3\).
    \item Since \(f_4\) is surjective, there exists \(a_4 \in A_4\) such that
    \(f_4 a_4 = b_4\), and furthermore \(d f_4 a_4 = d b_4 = f_3 a_3\).
    \item By commutativity, \(f_3 d a_4 = d f_4 a_4 = f_3 a_3\).
    \item Rearranging, \(f_3(a_3 - d a_4) = 0\).
    \item Since \(f_3\) is injective, \(a_3 = d a_4\).
    \item Applying \(d\) to both sides and using injectivity,
    \(d a_3 = d^2 a_4 = 0\), as desired. \pog
  \end{itemize}
\iffalse
  Let's show \(f_2\) is surjective. Pick \(b_2 \in B_2\). Note that
  \[
    d b_2 = f_1 a_1
  \]
  for some \(a_1\). Also note \(d a_1 = 0\) because
  \[
    f_0 d a_1 = d f_1 a_1 = d d b_2 = 0.
  \]
  By exactness, \(a_1 = d a_2\) for some \(a_2 \in A_2\).
  %We might like that f_2 a_2 = b_2, but we don't know it
  Note that \(d f_2 a_2 = f_1 d a_2 = f_1 a_1 = d b_2\).
  Consider \(b_2 - f_2 a\). This goes to \(0\) under \(d\).
  So we can find \(b_3 \in B_3\) such that \(d b_3 = b_2 - f_2 a_2\).
  Since \(f_3\) is an isomorphism, we can lift this to an \(a_3\) such that
  \(f_3 a_3 = b_3\) and now
  \[
    f_2 d a_3 = d f_3 a_3 = b_2 - f_2 a_2.
  \]
  Thus,
  \(b_2 = f_2(a_2 + d a_3)\).
\fi
\end{proof}

\begin{definition}
  A \vocab{short exact sequence of chain complexes} is a diagram
  \[
    \begin{tikzcd}
    	0 \ar[r] &
    	A_* \ar[r] &
    	B_* \ar[r] &
    	C_* \ar[r] &
    	0
    \end{tikzcd}
  \]
  or equivalently
  \[
    \begin{tikzcd}
    	 &
    	0 \ar[d] &
    	0 \ar[d] &
    	0 \ar[d] &
    	\\
    	\cdots \ar[r] &
    	A_{n+1} \ar[r] \ar[d] &
    	A_n     \ar[r] \ar[d] &
    	A_{n-1} \ar[r] \ar[d] &
    	\cdots \\
    	\cdots \ar[r] &
    	B_{n+1} \ar[r] \ar[d] &
    	B_n     \ar[r] \ar[d] &
    	B_{n-1} \ar[r] \ar[d] &
    	\cdots \\
    	\cdots  \ar[r] &
    	C_{n+1} \ar[r] \ar[d] &
    	C_n     \ar[r] \ar[d] &
    	C_{n-1} \ar[r] \ar[d] &
    	\cdots \\
    	&
    	0 &
    	0 &
    	0 &
    \end{tikzcd}
  \]
  where the rows are chain complexes and the vertical maps are
  short exact sequences of abelian groups.
\end{definition}

\begin{example}
  Suppose \(A \subseteq X\) is an inclusion of topological spaces. Then
  \[
    \begin{tikzcd}
    	0 \ar[r] &
    	S_*(A) \ar[r] &
    	S_*(X) \ar[r] &
    	S_*(X)/S_*(A) \ar[r] &
    	0
    \end{tikzcd}
  \]
  is a short exact sequence of chain complexes.
  This is easy to check because there is a natural injection
  \(S_*(A) \to S_*(X)\) and a natural surjection \(S_*(X) \to S_*(X)/S_*(A)\).
\end{example}

\begin{theorem}
  If
  \[
    \begin{tikzcd}
    	0 \ar[r] &
    	A_* \ar[r, "f"] &
    	B_* \ar[r, "g"] &
    	C_* \ar[r] &
    	0
    \end{tikzcd}
  \]
  is a short exact sequence of chain complexes, then there is
  a ``long'' exact sequence of homology groups
  \[
    \begin{tikzcd}
      \cdots \ar[r] &
    	H_{n+1}(B) \ar[r, "H_{n+1}(g)"] &
    	H_{n+1}(C) \ar[r] &
    	H_n(A)     \ar[r, "H_{n}(f)"] &
    	H_n(B)     \ar[r, "H_{n}(g)"] &
    	H_n(C)     \ar[r] &
    	H_{n-1}(A) \ar[r, "H_{n-1}(f)"] &
    	H_{n-1}(B) \ar[r] &
    	\cdots
    \end{tikzcd}
  \]
  where the interesting maps are the ones \(H_{n+1}(C) \to H_n(A)\).
\end{theorem}
In particular, there is a relationship between the homology of \(C\) and \(A\),
with the degree lowered by \(1\).
This is a version of what is called the ``snake lemma''.

The proof is again by diagram chasing, so we will defer it to the homework.
It is also proved in
\href{https://www.youtube.com/watch?v=etbcKWEKnvg}{this clip} from a movie.

\begin{corollary}
  Suppose \(A \subseteq X\) is an inclusion of spaces.
  Then there is a long exact sequence
  \[
    \begin{tikzcd}[column sep=small]
    	\cdots \ar[r] &
    	H_{n+1}(X) \ar[r] &
    	H_{n+1}(X, A) \ar[r] &
    	H_n(A) \ar[r] &
    	H_n(X) \ar[r] &
    	H_n(X, A) \ar[r] &
    	H_{n-1}(A) \ar[r] &
    	H_{n-1}(X) \ar[r] &
    	\cdots
    \end{tikzcd}
  \]
\end{corollary}
The point of this is that there exists a long exact sequence, even though
the degree lowering maps are mysterious and unknown. In particular,
to express this relationship, we have to use all of the homologies at once, as opposed to a
homology in a fixed degree.

\begin{question}
   How do we compute \(H_n(X, A)\)?
\end{question}

Simple case: If \(b \in X\), i.e.\ a pair \((X, \set{b})\),
what is \(H_m(X, \set{b})\)?

Consider
\[
  \begin{tikzcd}
  	S_*(\set{b}) \ar[r] &
  	S_*(X) \ar[r] &
  	S_*(X)/S_*(\set{b})
  \end{tikzcd}
\]
Note for all \(m \geq 0\), the quotient \(S_m(X) / S_m(\set{b})\) is
the quotient of the free abelian group generated by the \(m\)-simplices
of \(X\) by the group generated by the single \(m\)-simplex
\[
  \begin{tikzcd}
  	\Delta^m \ar[r] &
  	* \ar[r, "b"] &
  	X
  \end{tikzcd}
\]
This \(m\)-simplex is easy to calculate the boundary of.
For \(m > 0\), the homology \(H_m(X) \iso H_m(X, \set{b})\).
When \(m = 0\), we know from the problem set that \(H_0(X) \iso \ZZ \pi_0 X\).
Indeed, there is a direct sum decomposition
\[
  H_0(X) \iso \ZZ \oplus H_0(X, \set{b})
\]
where the summand \(\ZZ\) corresponds to the path component of \(b\).



\end{document}
