\documentclass{standalone}
\usepackage{chez}

\begin{document}
\chapter{October 21, 2020}

\begin{definition}
  Suppose \(C_*\) and \(D_*\) are chain complexes of \(R\)-modules.
  A chain map \(f \colon C_* \to D_*\) is a \vocab{quasi-isomorphism}
  if \(H_q(f)\) is an isomorphism for all \(q\).
  Then, one says that \(C_*\) and \(D_*\) are \vocab{quasi-isomorphic}.
\end{definition}

\begin{definition}
  Suppose \(M\) is an \(R\)-module.
  Then a \vocab{free resolution} of \(M\) is
    a chain complex \(C_*\) of free \(R\)-modules and
    a quasi-isomorphism \(C_* \to M\) where
      we think of \(M\) as a chain complex concentrated in degree \(0\).
\end{definition}

\begin{example}
  Suppose
  \begin{align*}
    C_* &= \Big(
    \begin{tikzcd}[ampersand replacement=\&]
    	\cdots \ar[r] \&
    	0 \ar[r] \&
    	0 \ar[r] \&
    	\ZZ \ar[r, "5"] \&
    	\ZZ \ar[r] \&
    	0 \ar[r] \&
    	\cdots
    \end{tikzcd}
    \Big) \\
    D_* &= \Big(
    \begin{tikzcd}[ampersand replacement=\&]
    	\cdots \ar[r] \&
    	0 \ar[r] \&
    	0 \ar[r] \&
    	0 \ar[r] \&
    	\ZZ/5\ZZ \ar[r] \&
    	0 \ar[r] \&
    	\cdots
    \end{tikzcd}
    \Big)
  \end{align*}
  Then the following chain map is a quasi-isomorphism:
  \[
    \begin{tikzcd}
      \cdots \ar[r] &
        0 \ar[r] \ar[d] &
        \ZZ \ar[r, "5"] \ar[d] &
        \ZZ \ar[r] \ar[d] &
        0 \ar[r] \ar[d] &
        \cdots \\
      \cdots \ar[r] &
        0 \ar[r] &
        0 \ar[r] &
        \ZZ/5\ZZ \ar[r] &
        0 \ar[r] &
        \cdots
    \end{tikzcd}
  \]
  where the map \(\ZZ \to \ZZ/5\ZZ\) is the natural quotient map.
  Moreover, \(C_*\) is a free resolution of \(\ZZ/5\ZZ\).
\end{example}

\begin{example}
  If \(R\) is a field, then every module is its own free resolution.
\end{example}

\begin{example}
  Suppose \(\RR = \ZZ\) and \(M = \ZZ/3\ZZ \oplus \ZZ \oplus \ZZ/2\ZZ\).
  Then we have the free resolution
  \[
    \begin{tikzcd}
    	\cdots \ar[r] &
    		0 \ar[r] \ar[d] &
    		\ZZ \oplus \ZZ \ar[r] \ar[d] &
    		\ZZ \oplus \ZZ \oplus \ZZ \ar[r] \ar[d] &
    		0 \ar[r] \ar[d] &
    		0 \ar[r] \ar[d] &
    		\cdots \\
    	\cdots \ar[r] &
    		0 \ar[r] &
    		0 \ar[r] &
    		\ZZ/3\ZZ \oplus \ZZ \oplus \ZZ/2\ZZ \ar[r] &
    		0 \ar[r] &
    		0 \ar[r] &
    		\cdots
    \end{tikzcd}
  \]
  where the \(\ZZ \oplus \ZZ \to \ZZ \oplus \ZZ \oplus \ZZ\) map
  maps \((1, 0) \mapsto (3, 0, 0)\) and \((0, 1) \mapsto (0, 0, 2)\).
\end{example}

It turns out that we can always find a two term free resolution by
taking a surjection with the degree \(0\) group,
e.g.\ \(\ZZ \oplus \ZZ \oplus \ZZ \to \ZZ/3\ZZ \oplus \ZZ \oplus \ZZ/2\ZZ\)
in the previous example,
and then use the degree \(1\) group to describe the relations.

Note that free resolutions are not unique! In particular,
\[
  \begin{tikzcd}[row sep=0.05ex]
  	\ZZ \ar[r] &
  		\ZZ \oplus \ZZ \oplus \ZZ \ar[r] &
  		\ZZ \oplus \ZZ \oplus \ZZ \\
    1 \ar[r, mapsto] & (0, 0, 1) \\
    & (1, 0, 0) \ar[r, mapsto] & (3, 0, 1) \\
    & (0, 1, 0) \ar[r, mapsto] & (0, 0, 2) \\
    & (0, 0, 1) \ar[r, mapsto] & (0, 0, 0)
  \end{tikzcd}
\]
is also a free resolution of \(\ZZ/3\ZZ \oplus \ZZ \oplus \ZZ/2\ZZ\).
However, it is not ``minimal''.

\begin{example}
  If \(R = \QQ[t]/t^2\) and
    \(M\) is the module \(\QQ\) where \(t\) acts by \(0\),
    then this has a free resolution, but the smallest possible one is infinite.
\end{example}

\begin{theorem}[Fundamental theorem of homological algebra]
  Suppose \(N\) and \(M\) are \(R\)-modules.
  Let
  \[
    \begin{tikzcd}
    	\cdots \ar[r] &
    		F_2 \ar[r] \ar[d] &
    		F_1 \ar[r] \ar[d] &
    		F_0 \ar[d, "\eps_N"] \\
    	\cdots \ar[r] &
    		0 \ar[r] &
    		0 \ar[r] &
    		N
    \end{tikzcd}
    \qquad \text{and} \qquad
    \begin{tikzcd}
    	\cdots \ar[r] &
    		E_2 \ar[r] \ar[d] &
    		E_1 \ar[r] \ar[d] &
    		E_0 \ar[d, "\eps_M"] \\
    	\cdots \ar[r] &
    		0 \ar[r] &
    		0 \ar[r] &
    		M
    \end{tikzcd}
  \]
  be two free resolutions of \(N\) and \(M\) respectively.
  Then any \(R\)-module map \(f \colon N \to M\) lifts to
  a chain map \(f_* \colon F_* \to E_*\).
  Furthermore, \(f_*\) is unique up to chain homotopy.
\end{theorem}

\begin{example}
  Let \(R = \ZZ\), \(N = \ZZ/2\ZZ\), and \(M = \ZZ/6\ZZ\).
  Consider \(f \colon \ZZ/2\ZZ \to \ZZ/6\ZZ\) mapping \(1 \mapsto 3\).
  We have the free resolutions with the maps
  \[
    \begin{tikzcd}[row sep=small,
                   column sep=small,
                   background color=green!6]
    	F_* = \Big( \cdots \ar[rr] &&
    		0 \ar[rr] \ar[ld] &&
    		\ZZ \ar[rr, "2"] \ar[ld, "1"'] &&
    		\ZZ \ar[rr] \ar[ld, "3"'] \ar[dd] &&
    		0 \ar[rr] \ar[ld] &&
    		\cdots \Big) \\
    	E_* = \Big( \cdots \ar[r] &
    		0 \ar[rr] &&
    		\ZZ \ar[rr, "6"'] &&
    		\ZZ \ar[rr, crossing over] \ar[d] &&
    		0 \ar[rr] &&
    		\cdots \Big) \\
    	&& && & \ZZ/6\ZZ & \ZZ/2\ZZ \ar[l, "f"']
    \end{tikzcd}
  \]
  This is a chain map \(f_*\), where
  \[
    \begin{tikzcd}[row sep=tiny]
      \mathllap{H_0(f_*) \colon} H_0(F_*) \ar[r] \ar[d, symbol=\iso] &
        H_0(E_*) \ar[d, symbol=\iso] \\
        \ZZ/2\ZZ \ar[r, "f"] & \ZZ/6\ZZ
    \end{tikzcd}
  \]
\end{example}

This theorem says that if we want to understand maps between modules,
it suffices to understand the maps between their free resolutions
up to chain homotopy.
In particular, they contain the same data,
and the map between the modules can be recovered by taking \(H_0\).

\begin{proof*}{Sketch}
  We can build up a map between the free resolutions inductively,
  checking at every stage that there is only one choice up to chain homotopy.
  To start producing a chain map, if we have
  \[
    \begin{tikzcd}
      F_0 \arrow[d, dashed, "f_0"] \arrow[r, "\eps_N"] &
        N \arrow[d, "f"] \\
      E_0 \arrow[r, "\eps_M"] &
        M
    \end{tikzcd}
  \]
  we want to produce the dashed \(f_0\) map.
  Since \(F_0\) is free, say on \(S_0\), for each \(s_0 \in S_0\),
  we can define \(f_0(s_0)\) to be any element in \(E_0\) such that
  \[
    \eps_M(f_0 s_0) = f (\eps_N s_0).
  \]
  This is possible because \(\eps_M\) is surjective,
  because \(M\) is the homology of \(E_*\),
  which means \(M\) is the quotient of
  \(E_0\) mod the image of the \(E_1 \to E_0\) map.
  
  This gives the diagram
  \[
    \begin{tikzcd}
      \ker \eps_M \arrow[d, "g_0"] \arrow[r] &
      F_0 \arrow[d, "f_0"] \arrow[r, "\eps_N"] &
        N \arrow[d, "f"] \\
      \ker \eps_N \arrow[r] &
      E_0 \arrow[r, "\eps_M"'] &
        M
    \end{tikzcd}
  \]
  where the map \(g_0\) is unique because \(\ker \eps_M \to F_0\)
  and \(\ker \eps_N \to E_0\) are just projections,
  so \(f_0\) determines \(g_0\).

  Now we wish to produce the dashed map in the diagram
  \[
    \begin{tikzcd}
      F_1 \ar[r] \ar[d, dashed, "f_1"] &
      \ker \eps_M \arrow[d, "g_0"] \arrow[r] &
      F_0 \arrow[d, "f_0"] \arrow[r, "\eps_N"] &
        N \arrow[d, "f"] \\
      E_1 \ar[r] &
      \ker \eps_N \arrow[r] &
      E_0 \arrow[r, "\eps_M"'] &
        M
    \end{tikzcd}
  \]
  However, by exactness of the free resolutions,
  we know that the \(F_1 \to \ker \eps_N\) and \(E_1 \to \ker \eps_M\) maps
  are surjective, so we can apply the same argument as before.
\end{proof*}

\begin{adhoctheorem}{Definition Sketch}
  Let \(R\) be a commutative ring, and \(\cchRmod\) be the category
  of chain complexes of \(R\)-modules.
  The \vocab{derived category} of \(R\), denoted \(\der(R)\),
  is the category obtained from \(\cchRmod\) by formally inverting
  all quasi-isomorphisms.

  In other words, the objects of \(\der(R)\) are the objects of \(\cchRmod\),
  but there are many more morphisms than just chain maps.
  In particular, if \(f_* \colon C_* \to D_*\) is a chain map,
  then there is a formal inverse \(g \colon D_* \to C_*\) in \(\der(R)\).
\end{adhoctheorem}
The formal construction of \(\der(R)\) is beyond the scope of the class
because there are set-theoretic technicalities.

In algebraic topology, we don't actually care about \(\cchRmod\).
We only care about \(\der(R)\).

\begin{example}
  In \(\der(R)\), every object is isomorphic to
  a chain complex of free modules.
  Free resolutions are an example of this.
\end{example}

We will state facts about \(\der(R)\) and then
prove consequences of these facts that can be stated
without reference to \(\der(R)\).
The reason we do so is because it gives motivation for the statements
and their proofs.

\begin{example}[Tensor products in \(\der(R)\)]
  If \(C_*\) and \(D_*\) are two chain complexes,
  we define the \vocab{derived tensor product}
  \(C_* \otimes^{\mathbb L} D_*\) as follows:
  \begin{enumerate}[nosep]
    \item We replace \(C_*\) and \(D_*\) by quasi-isomorphic chain complexes
          of free modules \(C_*'\) and \(D_*'\).
    \item Take \(C_*' \otimes_R D_*'\).
    \item The result is well defined up to quasi-isomorphism
  \end{enumerate}
\end{example}










\end{document}
