\documentclass{standalone}
\usepackage{chez}

\begin{document}
\chapter{November 16, 2020}

Recall that if \(A^*\) and \(B^*\) are graded \(R\)-algebras, then
\(A^* \oplus B^*\) is also a graded algebra with multiplication given by
\((a, b)(a', b') = (aa', bb')\), with unit \((1, 1)\).

\begin{remark}
  This direct sum is the product of \(A^*\) and \(B^*\) in
  the category of graded \(R\)-modules.
\end{remark}

\begin{proposition}
  Suppose \(X\) and \(Y\) are topological spaces. Then
  \(H^*(X \amalg Y; R) \iso H^*(X; R) \oplus H^*(Y; R)\)
  as \(R\)-algebras.
\end{proposition}
\begin{proof}
  The inclusions \(X \hookrightarrow X \amalg Y\) and
                 \(X \hookrightarrow X \amalg Y\) induce maps
                 of graded \(R\)-algebras:
  \begin{gather*}
    H^*(X \amalg Y; R) \to H^*(X; R) \\
    H^*(X \amalg Y; R) \to H^*(Y; R).
  \end{gather*}
  This induces a map, by the universal property of a product in a category
  \[
    H^*(X \amalg Y; R) \to H^*(X; R) \oplus H(Y; R).
  \]
  This is automatically a map of graded \(R\)-algebras.
  To prove it is an isomorphism, we need to just check that it
  is an isomorphism of abelian groups.
\end{proof}



\subsection{Cohomology of wedge products}
Suppose \(X\) is a topological space with \(x \in X\), and
        \(Y\) is a topological space with \(y \in Y\).
Then recall that \(X \vee Y = X \amalg Y / (x \sim y)\).
Note that there is a quotient map
\[
  r \colon X \amalg Y \to X \vee Y.
\]
Using the long exact sequence of a pair, % ???
we see that \(H_q(X \amalg Y) \to H_q(X \vee Y)\)
is an isomorphism for \(q > 0\).
By the universal coefficient theorem,
\[
  H^q(X \vee Y; R) \to H^q(X \amalg Y; R)
\]
is also an isomorphism.

\begin{corollary}
  The quotient map
  \[
    X \amalg Y \overset{r}{\to} X \vee Y
  \]
  induces a map of graded \(R\)-algebras
  \[
    H^*(X \vee Y) \to H^*(X \amalg Y)
  \]
  which is an isomorphism in positive degrees.
\end{corollary}
In degree \(0\), it is an injection.

\begin{example}
  Consider \(S^2 \vee S^1 \vee S^1\).
  The cohomology ring \(H^*(S^2 \vee S^1 \vee S^1; \ZZ)\) injects into
  \[
    H^*(S^2; \ZZ) \oplus
    H^*(S^1; \ZZ) \oplus
    H^*(S^1; \ZZ)
  \]
  that is an isomorphism in positive degrees,
  and an injection in degree \(0\).
  In particular, the cohomology ring injects into
  \[
    \underbracket{\ZZ \oplus \ZZ \oplus \ZZ}_{\deg 0} \oplus
    \underbracket{\ZZ \oplus \ZZ}_{\deg 1} \oplus
    \underbracket{\ZZ \oplus}_{\deg 2},
  \]
  where the degree \(0\) component is generated by
    \((1_{S^2}, 0, 0)\), \((0, 1_{S^1}, 0)\), and \((0, 0, 1_{S^1})\);
  the degree \(1\) component is generated by
    \((0, x, 0)\) and \((0, 0, x)\); and
  the degree \(2\) component is generated by
    \((y, 0, 0)\),
  where \(x \in H^1(S^1; \ZZ)\) and \(y \in H^2(S^2; \ZZ)\) are generators.

  Then \(H^*(S^2 \vee S^1 \vee S^1; \ZZ)\) is the subring generated by
  \[
    \underbracket{(1_{S^2}, 1_{S^2}, 1_{S^2})}_{\deg 0},
    \underbracket{(0, x, 0),
                  (0, 0, x)}_{\deg 1},
    \underbracket{(y, 0, 0)}_{\deg 2},
  \]
  i.e.\
  \[
    H^*(S^2 \vee S^1 \vee S^1; \ZZ)
      \iso \ZZ \oplus \ZZ \oplus \ZZ \oplus \ZZ.
  \]
  All products of positive degree classes are trivial.
\end{example}

\begin{question}
  Is \(S^2 \vee S^1 \vee S^1\) homotopy equivalent to the torus \(T^2\)?
  Note that \(H_q(S^2 \vee S^1 \vee S^1; \ZZ) \iso H_q(T; \ZZ)\)
  for all \(q\).

  However, the cohomology ring of \(S^2 \vee S^1 \vee S^1\)
  has trivial positive degree products,
  but the cohomology ring of \(T^2\) is
  \(\ZZ[x, y, z]/(x^2, y^2, xy-z, z^2, xz, yz)\).
  In particular, \(xy = z\), even though \(x\) and \(y\) are both
  in positive degrees.

  Therefore, these spaces are not homotopy equivalent, or in other words
  \[
    S^2 \vee S^1 \vee S^1 \ncong T^2
  \]
  in \(\cHoTop\).
  Another way to think about this is that the diagonal maps
  \begin{gather*}
    (S^2 \vee S^1 \vee S^1) \to
      (S^2 \vee S^1 \vee S^1) \times (S^2 \vee S^1 \vee S^1) \\
    T^2 \to T^2 \times T^2
  \end{gather*}
  induce different maps in homology.
\end{question}


\section{Poincar\'e duality}
Suppose \(A\) is an abelian group and \(R\) is a ring.
Then there is a map
\[
  \ul\Hom_{\cAb}(A, R) \otimes_\ZZ A \overset{f}{\to} R
\]
adjoint, under the currying isomorphism, to the identity
\[
  \ul\Hom_\cAb(A, R) \overset{g}{\to}
  \ul\Hom_\cAb(A, R).
\]
In particular, this is the evaluation map
\(\varphi \otimes m \mapsto \varphi(m)\).

Similarly, if \(M\) is an \(R\)-module, there is a map
\[
  \ul\Hom_\cRmod(M, R) \otimes_R M \overset{f}{\to} R
\]
adjoint to the identity
\[
  \ul\Hom_\cRmod(M, R) \to \ul\Hom_\cRmod(M, R).
\]

\subsection{Kronecker pairing}
Suppose \(X\) is a topological space and \(R\) is a ring.
Then there is a pairing map
\[
  \angles{{-}, {-}} \colon H^q(X; R) \otimes_R H_q(X; R) \to R.
\]

To see this, note that a class in \(H^q(X; R)\) is represented by
a cocycle in \(S^q(X; R)\) modulo coboundaries,
i.e.\ functions \(\Sing_q(X) \to R\).
A class in \(H^q(X; R)\) is represented by
a formal \(R\)-linear combination of classes in \(\Sing_q(X)\),
which are cycles modulo boundaries.

We can take any function \(\Sing_q(X) \to R\) and evaluate it on
any formal \(R\)-linear combination of classes in \(\Sing_q(X)\).
In other words, we have a map
\[
  S^q(X; R) \otimes_R S_q(X; R) \to R,
\]
which extends to a chain map
\[
  S^*(X; R) \otimes_R S_*(X; R) \to R,
\]
or a map
\[
  H^*(X; R) \otimes_R H_*(X; R) \to R.
\]













\end{document}
