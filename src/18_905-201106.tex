\documentclass{standalone}
\usepackage{chez}

\begin{document}
\chapter{November 06, 2020}

\begin{proof}[\cref{thm:cohomology-ring}]
  We want to prove that \(H^*(X; R)\) is an
    associative, graded-commutative, unital ring.
  We will first make the cup product explicit.
  An element of \(H^p(X; R)\) is represented by a class
  \[
    f \in S^p(X; R) = \ul\Hom_{\cAb}(S_p(X), R),
  \]
  where \(f\) is a \vocab{cocycle},
  and the cohomology class represented by \(f\)
  does not change if we add a \vocab{coboundary} to \(f\).
  Here, cocycles and coboundaries correspond to cycles and boundaries
  in regular homology.

  Explicitly, an \(f \in S^p(X; R)\) is a function \(S_p(X) \to R\)
  that respects addition. This can be thought of as a function
  \(f \colon \Sing_p(X) \to R\).
  In particular, for every \(\sigma \colon \Delta^p \to X\),
  there is \(f(\sigma) \in R\).

  Now suppose that \(f \in S^p(X; R)\) and \(g \in S^q(X; R)\)
  are elements of the cohomology ring.
  The cup product of \(f\) and \(g\) is the element \(fg \in S^{p+q}(X; R)\)
  such that for each \(\sigma \colon \Delta^{p+q} \to X\), we have
  \[
    (fg)(\sigma) =
      (-1)^{pq} f(\sigma{\restriction}_{\Delta^p})
                g(\sigma{\restriction}_{\Delta^q}),
  \]
  where the restriction to \(\Delta^p\) is to
  the front \(p\)-face of \(\Delta^{p+q}\), and
  the restriction to \(\Delta^q\) is to
  the back \(q\)-face of \(\Delta^{p+q}\).

  \paragraph{Associativity}
  Suppose \(f \in S^P(X; R)\), \(g \in S^q(X; R)\), and \(h \in S^r(X; R)\)
  are cocycles, and \(\sigma \in \Sing_{p+q+r}(X)\) is an arbitrary simplex.
  Then
  \begin{align*}
    ((fg)h)(\sigma)
      &= (-1)^{(p+q)r} (fg)(\sigma{\restriction}_{\Delta^{p+q}})
                         h (\sigma{\restriction}_{\Delta^r}) \\
      &= (-1)^{(p+q)r} (-1)^{pq} f(\sigma{\restriction}_{\Delta^p})
                                 g(\sigma{\restriction}_{\Delta^q})
                                 h(\sigma{\restriction}_{\Delta^r}) \\
      &= (-1)^{pr + pq + qr} f(\sigma{\restriction}_{\Delta^p})
                             g(\sigma{\restriction}_{\Delta^q})
                             h(\sigma{\restriction}_{\Delta^r}),
  \end{align*}
  where the restriction to \(\Delta^p\) is the first \(p\)-simplex,
        the restriction to \(\Delta^q\) is the middle \(q\)-simplex, and
        the restriction to \(\Delta^r\) is the middle \(r\)-simplex.
  We can compare this to
  \begin{align*}
    (f(gh))(\sigma)
      &= (-1)^{p(q+r)}    f(\sigma{\restriction}_{\Delta^p})
                       (gh)(\sigma{\restriction}_{\Delta^{q+r}}) \\
      &= (-1)^{(p+q)r}           f(\sigma{\restriction}_{\Delta^p})
                       (-1)^{qr} g(\sigma{\restriction}_{\Delta^q})
                                 h(\sigma{\restriction}_{\Delta^r}) \\
      &= (-1)^{pr + pq + qr} f(\sigma{\restriction}_{\Delta^p})
                             g(\sigma{\restriction}_{\Delta^q})
                             h(\sigma{\restriction}_{\Delta^r}),
  \end{align*}
  which is the same expression.
  Therefore, the cup product is associative.
  

  \paragraph{Unitality}
  We claim that the unit \(1 \in H^0(X; R)\) can be represented by
  the function \(1 \colon \Sing_0(X) \to R\),
  mapping everything to \(1 \in R\).
  For any cocycle \(f \in S^p(X; R)\), we have
  \begin{align*}
    (f \cdot 1)(\sigma)
      &= (-1)^{p \cdot 0} f(\sigma{\restriction}_{\Delta^p})
                          1(\sigma{\restriction}_{\Delta^0}) \\
      &= f(\sigma) 1 = f(\sigma).
  \end{align*}


  \paragraph{Graded-commutativity}
  This property is a bit more involved.
  Suppose \(f \in S^p(X; R)\) and \(g \in S^q(X; R)\) are cocycles and
  let \(\sigma \colon \Delta^{p+q} \to X\).
  Then
  \begin{align*}
    (fg)(\sigma) &= (-1)^{pq} f(\sigma{\restriction}_{\Delta^p})
                              g(\sigma{\restriction}_{\Delta^q}) \\
    (gf)(\sigma) &= (-1)^{pq} g(\sigma{\restriction}_{\Delta^q})
                              f(\sigma{\restriction}_{\Delta^p}). \\
  \end{align*}
  However in \(fg\), the restriction to \(\Delta^p\) is the front \(p\)-face
                 and the restriction to \(\Delta^q\) is the back \(q\)-face,
  while in \(gf\), the restriction to \(\Delta^p\) is the back \(p\)-face
               and the restriction to \(\Delta^q\) is the front \(q\)-face.
  These are not obviously related.
  In particular, we want to prove that
  \((fg) - (-1)^{pq}(gf)\) is a coboundary.
  
  The key point is that there is an interesting chain map
  \[
    S_*(X) \to S_*(X)
  \]
  homotopic to the identity
  where in degree \(p\) it maps a simplex \(\sigma \colon \Delta^p \to X\)
  to \((-1)^{p(p=1)/2} \tilde \sigma\) where \(\tilde\sigma\) is the composite
  \[
    \begin{tikzcd}[row sep=0]
      \Delta^p \ar[r] &
        \Delta^p \ar[r, "\sigma"] &
        X \\
      \brackets{v_0: v_1: \dots: v_p} \ar[r, mapsto] &
        \brackets{v_p: v_{p-1}: \dots: v_0}
    \end{tikzcd}
  \]
  Note that the number of coordinate transpositions
  has the same parity as \(p(p+1)/2\).
  This is related to the sign that we introduce,
  which is related to other signs where when we do a reflection,
  we multiply by \(-1\).

  Theorem 3.11 in Hatcher uses this fact to prove that
  \((fg) - (-1)^{pq}(gf)\) is a coboundary.
\end{proof}

Note that if \(X\) is a topological space and \(R\) is a ring,
then \(H^*(X; R)\) is not just a graded ring, but a graded \(R\)-algebra.
This means that \(H^*(X; R)\) is an \(R\)-module and
the multiplication is \(R\)-linear, i.e.\ the cup product is a map
\[
  H^*(X; R) \otimes_R H^*(X; R) \to H^*(X; R).
\]


\subsection{Cohomology ring of a product space}
Suppose \(A^*\) and \(B^*\) are two graded \(R\)-algebras.
Consider \(A^* \otimes_R B^*\) as the tensor product of \(R\)-modules.
We claim that this has a tensor product graded \(R\)-algebra structure.
For homogeneous \(a, b, a', b'\), we define
\[
  (a' \otimes b') (a \otimes b) = (-1)^{\abs{b'}\abs{a}} (a' a \otimes b' b).
\]







%If \(X \to Y\) is a map of topological spaces,
%then there is an induced map \(H^*(Y; R) \to H^*(X; R)\) of rings.








\end{document}
