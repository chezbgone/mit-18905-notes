\documentclass{standalone}
\usepackage{chez}

\begin{document}
\chapter{November 02, 2020}

\section{Homology with coefficients}
One thing we may be wondering is why we care about homology with coefficients
if we can compute them from homology with integer coefficients.
Earlier, we talked about how in applied algebraic topology with huge datasets,
it is provably faster to compute homology with rational coefficients than
with integer coefficients.

We can also give some pure math reasons.
\begin{example}[Group homology]
  If \(G\) is a group, we can form
    a category \(BG\) of one object, and
    a simplicial set \(N(BG)\).
  We can then ask what \(H_q(N(BG); M)\) is for various
  \(q \in \ZZ\), groups \(G\), and coefficient groups \(M\).
  While we can calculate this explicitly, even in the integers,
  one problem is that the number of simplices grows exponentially.
  Therefore, while we do have an algorithm for computing them,
  there is in general no closed form.

  In particular, if \(G = \GL_n(\FF_q)\),
  then this is known for \(M = \QQ\),
  but for \(M = \ZZ\) it is an open problem.
\end{example}

\begin{example}[Configuration spaces]
  Let \(X\) be a topological space and \(k \in \ZZ\).
  Then \(\operatorname{Config}_k(X)\) is the subspace of \(X^{\times k}\)
  given by
  \[
    \set{(x_1, \dots, x_k) \in X^{\times k} \mid
      \text{\(x_i \neq x_j\) when \(i \neq j\)}
    }.
  \]
  Then the unordered configuration space is
  \(B_k(X) = \operatorname{Config}_k(X) / S_k\),
  where \(S_k\) is the symmetric group on the points.
  \begin{itemize}[nosep]
    \item \(H_q(B_k(\RR^n); \ZZ)\) are known, but
    \item \(H_q(B_k(R^2); \ZZ)\) are unknown.
  \end{itemize}
  In particular, in the 1980s,
  people proved explicit formulas for \(H_q(B_k(T^2); \FF_2)\),
  and in the 2010s, people proved formulas for \(H_q(B_k(T^2), \QQ)\).
\end{example}


\section{Cohomology}
Recall that if we have a topological space\(X\)
and an abelian group \(M\), then we can form
\(\ul\Hom_{\ZZ\text{-}\cat{mod}}(S_n(X), M)\), % chktex 35
which gives a chain \(\ul\Hom_{\der(\ZZ)}(S_*(X), M)\).
This is an object of \(\der(\ZZ)\),
so it has a well defined homology groups.
In particular, \(H^q(X; M) \coloneqq H_{-q}(\ul\Hom_{\der(\ZZ)}(S_*(X), M))\)
is the \vocab{\(q\)th cohomology group} of \(X\) with coefficients in \(M\).

\begin{definition}
  Let \(S^q(X; M) \coloneqq \ul\Hom_\cAb(S_q(X), M)\).
\end{definition}
Since \(\ul\Hom_\cAb({-}, M) \colon \cAb^\op \to \cAb\) is functorial,
the map \(\partial \colon S_q(X) \to S_{q-1}(X)\) induces a map
\(\partial \colon S^{q-1}(X; M) \to S^q(X; M)\).
From this we can form the \vocab{cochain complex}
\[
  S^*_{\text{Hatcher}}(X; M) = \Big(\begin{tikzcd}
  	\cdots \ar[r] &
  		0 \ar[r] &
  		S^0(X; M) \ar[r, "\partial"] &
  		S^1(X; M) \ar[r, "\partial"] &
  		S^2(X; M) \ar[r] &
  		\cdots
  \end{tikzcd}
  \Big).
\]
This is a chain complex concentrated in nonpositive degrees.
Then we define
\[
  H^q(X; M) = \frac{\ker (S^q(X; M) \to S^{q+1}(X; M))}
                   {\img (S^{q-1}(X; M) \to S^q(X; M))}.
\]

\begin{adhoctheorem}{Warning}<cochain-warning>
  We can also form
  \[
    S^*_{\text{Miller}}(X; M) = \Big(\begin{tikzcd}
      \cdots \ar[r] &
        0 \ar[r] &
        S^0(X; M) \ar[r, "-\partial"] &
        S^1(X; M) \ar[r, "\partial"] &
        S^2(X; M) \ar[r, "-\partial"] &
        S^3(X; M) \ar[r] &
        \cdots
    \end{tikzcd}
    \Big)
  \]
  where in \(S^*_{\text{Miller}}\),
  the map \(S^q(X; M) \to S^{q+1}(X; M)\) is \((-1)^{q+1} \partial\),
  where \(\partial\) is the map in \(S^*_{\text{Hatcher}}\).
\end{adhoctheorem}

Note that \(S^*_{\text{Hatcher}}(X; M) \iso S^*_{\text{Miller}}(X; M)\)
in \(\der(\ZZ)\), so they have the same cohomology groups.
However, they are not exactly the same chain complex.
Here, we will use \(S^*(X; M)\) to denote \(S^*_{\text{Miller}}(X; M)\).

\begin{remark}
  \(S^*(X; \FF_2)\) is the same in both conventions.
\end{remark}

Recall the universal coefficient theorem for cohomology:
\[
  H^q(X; M) \iso \Ext^1_\ZZ(H_{q-1}(X); M) \oplus \ul\Hom_\cAb(H_q(X), M).
\]
This shows that cohomology with coefficients in \(M\) can be computed in terms
of homology with coefficients in \(\ZZ\).

\begin{example}
  Suppose we want to calculate \(H^2(\RP^2; \FF_2)\).

  The first method is to use the cellular cochain complex.
  First, we have the chain complex
  \[
    \Ccell_*(\RP^2, \ZZ) \iso \Big(
      \begin{tikzcd}
      	\cdots \ar[r] &
      		0 \ar[r] &
      		\ZZ \ar[r, "2"] &
      		\ZZ \ar[r, "0"] &
      		\ZZ \ar[r] &
      		0 \ar[r] &
      		\cdots
      \end{tikzcd}
    \Big)
  \]
  This gives the cochain complex
  \begin{align*}
    \CCell^*(\RP^2, \ZZ) &\iso \Big(
      \begin{tikzcd}[ampersand replacement=\&, column sep=scriptsize]
      	\cdots \ar[r] \&
      		0 \ar[r] \&
      		\ul\Hom_\cAb(\ZZ, \FF_2) \ar[r, "2"] \&
      		\ul\Hom_\cAb(\ZZ, \FF_2) \ar[r, "0"] \&
      		\ul\Hom_\cAb(\ZZ, \FF_2) \ar[r] \&
      		0 \ar[r] \&
      		\cdots
      \end{tikzcd}\Big) \\
      &\iso \Big(
        \begin{tikzcd}[ampersand replacement=\&, column sep=large]
          \cdots \ar[r] \&
            0 \ar[r] \&
            \FF_2 \ar[r, "0"] \&
            \FF_2 \ar[r, "0"] \&
            \FF_2 \ar[r] \&
            0 \ar[r] \&
            \cdots
        \end{tikzcd}\Big)
  \end{align*}
  Therefore,
  \[
    H^q(\RP^2; \FF_2) = \begin{cases*}
      \FF_2 & \(q = 0, 1, 2\) \\[-1ex]
      0 & otherwise.
    \end{cases*}
  \]

  The second method we can use is to apply the universal coefficient theorem.
  We have
  \begin{align*}
    H^2(\RP^2; \FF_2) &\iso \Ext^1_\ZZ(H_q(\RP^2), \FF_2) \oplus
                            \ul\Hom_\cAb(H_2(\RP^2), F_2) \\
      &\iso \Ext^1_\ZZ(\ZZ/2\ZZ, \FF_2) \oplus \ul\Hom_\cAb(0, \FF^2) \\
      &\iso \Ext^1_\ZZ(\FF_2, \FF_2).
  \end{align*}
  To compute \(\Ext^1_\ZZ(\FF_2, \FF_2)\), we need to compute
  \begin{align*}
    \mathop{H_{-1}} \ul\Hom_{\der(\ZZ)}(\FF_2, \FF_2)
      &\iso \mathop{H_{-1}} \ul\Hom_{\der(\ZZ)}(\FF_2, \FF_2) \\
      &\iso \mathop{H_{-1}} \ul\Hom_{\der(\ZZ)}(\,
        \begin{tikzcd}[cramped, ampersand replacement=\&, column sep=scriptsize]
          \underset{\mathclap{\deg 1}}\ZZ \ar[r, "2"] \&
          \underset{\mathclap{\deg 0}}\ZZ
        \end{tikzcd}\,,
        \FF_2
      ) \\
      &\iso \mathop{H_{-1}} (\,
        \begin{tikzcd}[cramped, ampersand replacement=\&, column sep=scriptsize]
          \underset{\mathclap{\deg 0}}\FF \ar[r, "2"] \&
          \underset{\mathclap{\deg {-1}}}\FF
        \end{tikzcd}
      \,) \\
      &\iso \FF_2.
  \end{align*}
\end{example}

\begin{definition}
  Let \(H^*(X; M) \iso \bigoplus_q H^q(X; M)\).
\end{definition}

\begin{question}
  How does \(H^*({-}; M)\) interact with diagonal maps?
\end{question}

If \(X \to Y\) is a map in \(\cTop\) and
   \(M\) is an abelian group,
then there is a natural map \(H^q(Y; M) \to H^q(X; M)\).
In particular, if \(X\) is any topological space,
then there is a natural map \(H^q(X \times X; M) \to H^q(X; M)\)
induced from the diagonal map.
Taking the direct sum over all \(q\), we get a map
\[
  H^*(X \times X; M) \to H^*(X; M).
\]

\begin{remark}
  Suppose \(R\) is a ring.
  Then if \(H^*(X; R)\) is a free \(R\)-module,
  \[
    H^*(X \times X; R) \iso H^*(X; R) \otimes_R H^*(X; R).
  \]
  This is similar to the result we got for regular homology.
\end{remark}

In particular, for any spaces \(X\) and \(Y\) and
any ring \(R\), we will construct a map
\[
  H^*(X; R) \otimes_R H^*(Y; R) \to H^*(X \times Y; R).
\]
In the case where \(H_*(X; R)\) and \(H_*(Y; R)\) are free,
this is an isomorphism.

\begin{corollary}
  If \(X\) is a topological space and \(R\) is a ring,
  then there is a natural multiplication
  \[
    H^*(X; R) \otimes_R H^*(X; R) \hookrightarrow H^*(X; R)
  \]
  which is the composite
  \[
    H^*(X; R) \otimes_R H^*(X; R) \to H^*(X \times X; R)
      \xrightarrow{H^q(\Delta; R)} H^*(X; R).
  \]
\end{corollary}

This means that we a structure \(H^*(X; R)\) that has both
an addition and multiplication operation.
It turns out that multiplication is associative and commutative,
so we can make this into a ring!





\end{document}
