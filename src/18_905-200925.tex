\documentclass{standalone}
\usepackage{chez}

\begin{document}
\chapter{September 25, 2020}

Let's give some more intuition about the dimension-lowering connecting map
in the long exact sequence.

Let \(A \subseteq X\) be a pair of spaces such that
\[
  H_m(A, X) \iso H_m(X/A, *) \iso H_m(X/A)
\]
for \(m > 0\). We have the long exact sequence
\[
  \begin{tikzcd}
    \cdots \ar[r] &
    H_m(A) \ar[r] &
    H_m(X) \ar[r] &
    H_m(X, A) \iso H_m(X/A) \ar[r, "\partial"] &
    H_{m-1}(A) \ar[r] &
    H_{m-1}(X) \ar[r] &
    \cdots
  \end{tikzcd}
\]
\begin{question}
  What is the geometric intuition of \(\partial\)?
\end{question}
Consider some class \(c \in S_m(X)\) not in \(Z_m(X)\),
i.e.\ \(\partial c \neq 0\).
Suppose that the image of \(c\) under the quotient map
\(S_m(X) \to S_m(X/A)\) is a cycle. Then \(\partial c \in S_{m-1}(X)\)
is actually in the image of \(S_{m-1}(A) \hookrightarrow S_{m-1}(X)\),
and \(\partial c \in Z_{m-1}(A)\).
Therefore, \(\partial c\) represents and element of \(H_{m-1}(A)\).

\begin{example}
  Suppose \(X = [0, 1]\) and \(A = \set{0, 1}\). Then \(X/A \iso S^1\).
  Consider \(\sigma \colon \Delta^1 \to X \in S_1(X)\) where
  \(\sigma \colon e_0 \mapsto 0, e_1 \mapsto 1\).

  In \(S_*(X)\), \(\partial \sigma = \set{1} - \set{0} \neq 0 \in S_0(X)\).
  However, under the quotient \(X \to X/A\), \(\sigma\) is sent to an element
  in \(Z_1(X/A)\). Therefore, \(\sigma\) represents a class in
  \(H_1(X, A) \iso H_1(S^1)\), but its boundary
  \(\partial \sigma = \set{1} - \set{0}\) is a class in \(H_0(A)\).
  Symbolically,
  \begin{align*}
    \cdots \to H_1(S^1) &\to H_0(\set{0, 1}) \to \cdots \\[-1ex]
    \ZZ &\to \ZZ \oplus \ZZ \\[-1ex]
    \sigma &\to (-1, 1).
  \end{align*}
\end{example}

\section{Excision}
Recall that there is a chain homotopy \(T\) from \(\$\) to \(1_{S_*(X)}\).
\begin{corollary}
  There is a natural chain homotopy \(T_k\) from \(\$^k\) to \(1_{S_*(X)}\).
\end{corollary}
\begin{corollary}
  If \(\sigma \in Z_n(X)\), then \(\$^k \sigma\) is equivalent to \(\sigma\)
  modulo boundaries.
\end{corollary}

We will now prove the locality principle.
\begin{theorem*}[\hyperref[thm:locality]{Locality principle}]
  Let \(\mathcal A\) be a cover of \(X\).
  Then \(S_*^{\mathcal A}(X) \subseteq S_*(X)\)
  induces an isomorphism \(H_n^{\mathcal A}(X) \leftrightarrow H_n(X)\).
\end{theorem*}
\begin{proof}
  First we will show surjectivity. Suppose we have some \(c \in Z_n(X)\).
  We want to find an \(\mathcal A\)-small element of \(Z_n(X)\) equivalent
  to \(c\) modulo boundaries. We just use \(\$^k c\) for a sufficiently large
  \(k\), and the Lebesgue covering lemma tells us this works.

  To show injectivity, we can show that the kernel is \(0\).
  In particular, suppose \(c \in Z_n^{\mathcal A}(X)\) is
  an \(\mathcal A\)-small cycle, and \(c = \partial b\) for some
  \(b \in S_{n+1}(X)\). We wish to find some \(b' \in S_{n+1}^{\mathcal A}(X)\)
  such that \(\partial b' = c\). We can select \(k\) such that \(\$^k b\) is
  \(\mathcal A\)-small. Note that
  \begin{align*}
    \partial(\$^k b) - c &= \partial ((\$^k - 1_{S_{n+1}(X)})(b)) \\
      &= \partial((\partial T_k + T_k \partial) b) \\
      &= \partial \partial T_k b + \partial (T_k \partial b) \\
      &= 0 + \partial (T_k c).
  \end{align*}
  Rearranging,
  \[
    c = \partial(\$^k b - T_k c).
  \]
  We finish by the following:
  \begin{lemma*}
    If \(c \in S_n(X)\) is \(\mathcal A\)-small, then
    \(T_k c \in S_{n+1}(X)\) is also \(\mathcal A\)-small.
  \end{lemma*}
  \begin{proof*}{Lemma proof}
    Without loss of generality assume that
    \(c = \sigma\) is an \(\mathcal A\)-small \(n\)-simplex.
    We can represent it as a composite
    \(\Delta^n \overset{\sigma'}\to A \overset{i}\to X\)
    where \(A \in \mathcal A\), where
      \(\sigma'\) denotes the map \(\Delta^n \to A\) and
      \(i\) is the inclusion of \(A\) into \(X\).
    Note that \(\sigma = S_n(i)(\sigma')\), so
    \(T_k \sigma = (S_{n+1}(i))(T_k \sigma')\)
    by the naturailty of \(T_k\).
    We know that \(T_k \sigma\) is the image of
    some \(T_k \sigma' \in S_{n+1}(A)\), so
    \(T_k \sigma\) is \(\mathcal A\)-small.
  \end{proof*}
  This concludes the proof of the locality principle.
\end{proof}

Now we can prove excision.
\begin{theorem*}[Excision]
  Suppose \(U \subseteq A \subseteq X\) is excisive, i.e.\
  \(\ol U \subseteq \Int A\). Then the map \((X - U, A - U) \to (X, A)\)
  induces homology isomorphisms.
\end{theorem*}
\begin{proof}
  Note that the condition \(\ol U \subseteq \Int A\) is equivalent to
  \[
    \Int A \union \Int (X - U) = X.
  \]
  Let \(B \coloneqq X - U\). Then \(\set{A, B}\) is a cover of \(X\).
  We can rewrite \((X - U, A - U) = (B, A \intersect B)\), so we want to show
  that the inclusion
  \[
    S_*(B, A \intersect B) \hookrightarrow S_*(X, A)
  \]
  induces homology isomorphisms.
  Note the diagram of a map of short exact sequences of chain complexes
  \[
    \begin{tikzcd}
      0 \ar[r]                                                      &
        S_*(A) \ar[r] \ar[d, equal]                                 &
        S_*^{\mathcal A}(X) \ar[r], \ar[d, hook]                    &
        S_*^{\mathcal A}(X)/S_*^{\mathcal A}(A) \ar[r] \ar[d, hook] &
        0                                                           \\
      0 \ar[r]                                                      &
        S_*(A) \ar[r]                                               &
        S_*(X) \ar[r]                                               &
        \underset{\mathclap{= S_*(X)/S_*(A)}}{S_*(X, A)} \ar[r]     &
        0
    \end{tikzcd}
  \]
  We can obtain a map of long exact sequences of homology groups
  \[
    \begin{tikzcd}
      \cdots \ar[r]                                               &
      H_m(A) \ar[r] \ar[d]                                        &
      H_m^{\mathcal A}(X) \ar[r] \ar[d]                           &
      H_m(S_*^{\mathcal A}(X) / S_*(X)) \ar[r, "\partial"] \ar[d] &
      H_{m-1}(A) \ar[r] \ar[d]                                    &
      H_{m-1}^{\mathcal A}(X) \ar[r] \ar[d]                       &
      \cdots                                                      \\
      \cdots \ar[r]                                               &
      H_m(A) \ar[r]                                               &
      H_m(X) \ar[r]                                               &
      H_m(S_*(X) / S_*(X)) \ar[r, "\partial"]                     &
      H_{m-1}(A) \ar[r]                                           &
      H_{m-1}^{\mathcal A}(X) \ar[r]                              &
      \cdots
    \end{tikzcd}
  \]
  By locality, the \(H_m^{\mathcal A}(X) \to H_m(X)\) and
  \(H_{m-1}^{\mathcal A}(X) \to H_{m-1}(X)\) maps are isomorphisms.
  By the five-lemma,
  \[
    H_m(S_*^{\mathcal A}(X) / S_*(X)) \to H_m(X, A)
  \]
  is also an isomorphism. Therefore,
  \(S_*^{\mathcal A}(X) / S_*(A) \to S_*(X) / S_*(A)\)
  induces homology isomorphisms \(H_m(S_*(X)/S_*(A)) \iso H_m(X, A)\).
  Then observe
  \begin{align*}
    S_n^{\mathcal A}(X) / S_n(A) &= \frac{S_n(A) + S_n(B)}{S_n(A)} \\
      &= \frac{S_n(B)}{S_n(A) \intersect S_n(B)} \\
      &= \frac{S_n(B)}{S_n(A \intersect B)} \\
      &= \frac{S_n(X - U)}{S_n(A - U)}. \pog
  \end{align*}
\end{proof}

\begin{corollary}
  Let \((X, A)\) be a pair of spaces and let \(B \subseteq X\) be a subspace
  such that
  \begin{enumerate}[nosep]
    \item \(\overline A \subseteq \Int B\) and
    \item \(A \to B\) is a deformation retract.\label{item-def-retract}
  \end{enumerate}
  Then for all \(m\), \(H_m(X, A) \to H_m(X/A, *)\) is an isomorphism.
\end{corollary}
\begin{proof}
  Consider the commutative diagram in \(\cTop_2\)
  \[
    \begin{tikzcd}
      (X, A) \arrow[r, "i"] \arrow[d] &
        (X, B) \arrow[r, leftarrow, "j"] \arrow[d] &
        (X - A, B - A) \arrow[d, "k"] \\
      (X/A, *) \arrow[r, "\ol i"] &
        (X/A, B/A) \arrow[r, leftarrow, "\ol j"] &
        (X/A - *, B/A - *)
    \end{tikzcd}
  \]
  We can check that each labeled arrow induces a homology isomorphism:
  \begin{itemize}
    \item \(k\) is a homeomorphism in \(\cTop_2\).
    \item \(j\) is an excision
    \item \(i\) is a homology isomorphism by \cref{item-def-retract} and
      the homotopy invariance of homology.
    \item \(\ol j\) is an excision.
    \item \(\ol i\) is also a deformation detract, obtained from
    the given retract \(B \times I \to B\) by quotienting out by \(A\).
  \end{itemize}
  Since all of the maps along the edge induce an homology isomorphism,
  the map on the left also induces a homology isomorphism.
\end{proof}

This concludes the proof of all of the Eilenberg-Steenrod axioms.





\end{document}
