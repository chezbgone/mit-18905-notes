\documentclass{standalone}
\usepackage{chez}

\begin{document}
\chapter{December 02, 2020}
\section{Proof of Poincar\'e duality}
\subsection{Constructions}
So far, we have the constructions relating homology and cohomology:
\begin{description}
  \item[Cross products]
  \[
    \times \colon H^p(X; R) \otimes_R H^q(Y; R) \to H^{p+q}(X \times Y; R)
  \]
  
  \item[Cup products]
  \[
    \times \colon H^p(X; R) \otimes_R H^q(X; R) \to H^{p+q}(X \times X; R),
  \]
  which is the cross product combined with the diagonal map.

  \item[Kronecker pairing]
  \[
    H^p(X; R) \otimes_R H_p(X; R) \to R,
  \]
  which is taking some function on homology
  (i.e.\ and element of cohomology)
  and evaluating it.
\end{description}

\begin{definition}
  The \vocab{cap product} is the operation
  \[
    \cap \colon H^p(X; R) \otimes_R H_n(X; R) \to H_{n-p}(X; R).
  \]
  is defined by applying homology to the chain map
  \[
    S^p(X; R) \otimes_R S_n(X; R) \xrightarrow{\mathrm{AW}}
    S^p(X; R) \otimes_R S_p(X; R) \otimes_R S_{n-p}(X; R)
    \xrightarrow{\angles{-, -}} R \otimes_R S_{n-p}(X; R) \iso S_{n-p}(X; R),
  \]
  where \(\mathrm{AW}\) is the Alexander-Whitney map.
\end{definition}
This is a variation of the Kronecker pairing that doesn't require
the elements to come from a homology and cohomology of the same degree.

\begin{note}
  All of these constructions come from
  \begin{itemize}[nosep]
    \item the cross product,
    \item the diagonal map on a space, and
    \item the Kronecker pairing.
  \end{itemize}
\end{note}
Here are some properties of the cap product:
\begin{enumerate}
  \item \((a \cup b) \cap x = a \cap (b \cap x)\) and \(1 \cap x = x\).
    In other words, \(H_*(X; R)\) is a module for \(H^*(X; R)\).
  \item \(\angles{a \cup b, x} = \angles{a, b \cap x}\).
  \item Let \(\eps \colon H_0(X; R) \to R\) be
    the element adjoint to \(1 \in H^0(X; R)\).
    Then if \(b \in H^p(X; R)\) and \(x \in H_p(X; R)\), then
    \[
      \eps(b \cap x) = \angles{b, x}.
    \]
  \item(Projection formula) Given a map \(f \colon X \to Y\),
                                        \(b \in H^p(Y)\), and
                                        \(x \in H_n(X)\),
    \[
      H_*(f)(H^*(f)(b) \cap x) = b \cap (H_*(f)(x)).
    \]
\end{enumerate}

\begin{theorem}[Poincar\'e duality]
  Let \(M\) be a compact \(n\)-dimensional manifold
  equipped with an \(R\)-orientation for a PID \(R\).
  Then there is a unique \([M] \in H_n(M; R)\) that
  restricts to the local orientation in \(H_n(M; M - \set{x}) \otimes_\ZZ R\)
  for each point \(x \in M\), such that
  \begin{align*}
    H^p(M; R) &\to H_{n-p}(M; R) \\[-1ex]
      a &\mapsto a \cap [M]
  \end{align*}
  is an isomorphism for all \(p \in \ZZ\).
\end{theorem}
To understand this geometrically, consider the reverse map.
It takes a homology class and records the function on \(p\)th homology
that counts intersections with sign depending on the \(R\)-orientation
around each intersection.

We will prove this with \(R = \ZZ\) for the notation simplicity.
The more general proof is not any harder.
We will prove Poincar\'e duality by formulating
a more general statement that works for
compact subsets of arbitrary manifolds.

\subsection{Relative cap products}
Suppose \(A \subseteq X\) is a subspace of \(X\).
Then there is a \vocab{relative cap product}
\[
  H^p(X) \otimes H_n(X, A) \to H_{n-p}(X, A)
\]
making \(H_*(X, A)\) a module over \(H^*(X)\).
This is constructed via the following diagram.
\[
  \begin{tikzcd}[column sep=small]
    0 \ar[d] && 0 \ar[d] \\
    S^p(X) \otimes S_n(A) \ar[r] \ar[d] &
      S^p(A) \otimes S_n(A) \ar[r, "\cap"] &
      S_{n-p}(A) \ar[d] \\
    S^p(X) \otimes S_n(X) \ar[rr, "\cap"] \ar[d] &&
      S_{n-p}(X) \ar[d] \\
    S^p(X) \otimes S_n(X, A) \ar[rr, dashed] \ar[d] &&
      S_{n-p}(X, A) \ar[d] \\
    0 && 0
  \end{tikzcd}
\]
We have a short exact sequence of chain complexes on the left
from the definition of \(S_n(X, A) \iso S_n(X / A)\).
We then have the map \(S^p(X) \otimes S_n(A) \to S^p(A) \otimes S_n(A)\)
from the inclusion \(A \subseteq X\).
Then, the rectangle commutes by the projection formula.
Then, the dashed arrow exists by the universal property of a quotient.
Taking the homology of the dashed arrow gives the relative cap product.

\subsection{\v{C}ech cohomology}
Suppose \(K \subseteq X\) is a closed subset of a topological space \(X\).
By excision, we have
\[
  H_n(X, X-K) \iso H_n(U, U-K)
\]
for any open \(U\) containing \(K\).
Note that we have a relative cap product
\[
  H^p(U) \otimes H_n(U, U-K) \to H_{n-k}(U, U-K).
\]
Thus, \(H_*(X, X-K)\) is a module over \(H^*(U)\)
for every open \(U\) containing \(K\).

\begin{definition}
  The \(p\)th \vocab{\v{C}ech cohomology} of \(K\)
  is denoted \(\check{H}^p(K)\).
  An element \(x \in \check{H}^p(K)\) is an element of \(H^p(U)\)
  for some open \(U\) containing \(K\).
  If \(K \subseteq U \overset{}\subseteq V\), we let
  \(x \in H^p(V) \subseteq \check{H}^p(X)\) be equal to
  \[
    H^*(i)(x) \in H^p(U) \subseteq H^p(X).
  \]
\end{definition}
Informally, this tries to capture the cohomology rings of
arbitrarily small open subsets of \(K\).

\begin{remark}
  \(\check{H}^p(K)\) depends on how \(K\) sits in \(X\).
  For ``reasonable'' \(K \subseteq X\),
  we have \(\check{H}^*(K) \iso H^*(K)\),
  but this might fail for things like the Cantor set.

  An instance of ``reasonable'' is when \(K\) is a deformation retract
  of an open set.
\end{remark}

Note that \(\check{H}^*(K)\) is a ring.
Furthermore, \(H_n(X, X-K)\) is a module over \(\check{H}^p(K)\).

We now have
\[
  \cap \colon \check{H}^p(K) \otimes H_n(X, X-K) \to H_{n-p}(X, X-K).
\]
We will want to show that this is an isomorphism or perfect pairing
under various conditions on \(X \subseteq X\).
We will prove this by breaking \(X\) and \(K\) into smaller pieces
that we understand and reassembling the desired statement from Mayor-Vietoris
and the five lemma.

\begin{example}[Topologist's sine curve]
  Consider the graph of \(\sin(2\pi / x)\) for \(0 < x < 1\).
  Let \(K \subseteq \RR^2\) be the union of
    this curve,
    \(\set{0} \times [-1, 1]\), and
    \(\gamma\) defined to be a curve connecting \((0, -1)\) and \((1, 0)\)
    such that it doesn't intersect the rest of the curves.

  This looks like a circle,
  but it gets really complicated near the \(y\)-axis.

  We have
  \[
    H^*(K) \iso \begin{cases}
      \ZZ & \deg = 0 \\[-1ex]
      0 & \text{otherwise}.
    \end{cases}
  \]
  On the other hand,
  \[
    \check H^*(K) \iso H^*(S^1) \iso \begin{cases*}
      \ZZ & \text{\(\deg = 0\) and \(\deg = 1\)} \\[-1ex]
      0 & otherwise.
    \end{cases*}
  \]
\end{example}





\end{document}
