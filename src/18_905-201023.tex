\documentclass{standalone}
\usepackage{chez}

\begin{document}
\chapter{October 23, 2020}

If two chain complexes are isomorphic in \(\der(R)\),
then they have isomorphic homology \(R\)-modules.

\begin{fact}
  If at least one of \(C_*\) and \(D_*\) are
  chain complexes of free \(R\)-modules,
  then \(C_* \otimes_R^{\mathbb L} D_* \iso C_* \otimes_R D_*\).
\end{fact}
Since we haven't described the derived category rigorously,
we can't prove this. However, we will show some consequences of this fact.

\begin{definition}
  If \(M\) and \(N\) are two \(R\)-modules,
  then the \vocab{\(\Tor\)} functor is defined to be
  \(\Tor_i(M, N) \coloneqq H_i(M \otimes^{\mathbb L}_R N)\),
  where \(M\) and \(N\) are viewed as
  chain complexes concentrated in degree \(0\).
\end{definition}
Soon we will prove that \(\Tor_i(M, N)\) are well defined.
For now, let's do some calculations of them.

\begin{example}
  Suppose \(R = \ZZ\). Then we can compute
  \(\ZZ/2\ZZ \otimes^{\mathbb L}_\ZZ \ZZ/4\ZZ\)
  by replacing either \(\ZZ/2\ZZ\) or \(\ZZ/4\ZZ\) with their free resolution,
  and then take the tensor product normally.

  We can compute this by resolving \(\ZZ/2\ZZ\), i.e.\ replacing it with
  its free resolution.
  \begin{align*}
    \ZZ/2\ZZ \otimes^{\mathbb L} \ZZ/4\ZZ
      &\iso \Big(
        \begin{tikzcd}[ampersand replacement=\&,
                       cramped, column sep=scriptsize]
          \ZZ \ar[r, "2"] \& \ZZ
        \end{tikzcd}
      \Big) \otimes \ZZ/4\ZZ \\
      &\iso \Bigg(\;
        \begin{tikzcd}[ampersand replacement=\&,
                       cramped, column sep=scriptsize, row sep=0ex]
          \cdots \ar[r] \&
          0 \ar[r] \&
          \ZZ/4\ZZ \ar[r, "2"] \&
          \ZZ/4\ZZ \ar[r] \&
          0 \ar[r] \&
          \cdots \\
          \& \& \deg 1 \& \deg 0
        \end{tikzcd}
      \;\Bigg)
  \end{align*}
  Then we have
  \begin{align*}
    \Tor_0(\ZZ/2\ZZ, \ZZ/4\ZZ) &= \frac{\ZZ/4\ZZ}{\ZZ/2\ZZ} \iso \ZZ/2\ZZ \\
    \Tor_1(\ZZ/2\ZZ, \ZZ/4\ZZ) &= \frac{\ZZ/2\ZZ}{0} \iso \ZZ/2\ZZ,
  \end{align*}
  and \(\Tor_i = 0\) for all other \(i\).

  We can also compute it by resolving \(\ZZ/4\ZZ\):
  \begin{align*}
    \ZZ/2\ZZ \otimes^{\mathbb L} \ZZ/4\ZZ
      &\iso \ZZ/2\ZZ \otimes \Big(
        \begin{tikzcd}[ampersand replacement=\&,
                       cramped, column sep=scriptsize]
          \ZZ \ar[r, "4"] \& \ZZ
        \end{tikzcd}
      \Big) \\
      &\iso \Big(
        \begin{tikzcd}[ampersand replacement=\&,
                       cramped, column sep=scriptsize]
          \ZZ/4\ZZ \ar[r, "4"] \& \ZZ/4\ZZ
        \end{tikzcd}
      \Big)
  \end{align*}
  Then we have
  \begin{align*}
    \Tor_0(\ZZ/2\ZZ, \ZZ/4\ZZ) &= \frac{\ZZ/2\ZZ}{0} \iso \ZZ/2\ZZ \\
    \Tor_1(\ZZ/2\ZZ, \ZZ/4\ZZ) &= \frac{\ZZ/2\ZZ}{0} \iso \ZZ/2\ZZ,
  \end{align*}
  and \(\Tor_i = 0\) for all other \(i\).

  We can also compute this in some more inefficient ways.
  We can resolve both sides:
  \begin{align*}
    \ZZ/2\ZZ \otimes^{\mathbb L} \ZZ/4\ZZ
      &\iso \Big(
        \begin{tikzcd}[ampersand replacement=\&,
                       cramped, column sep=scriptsize]
          \ZZ\fgen{a} \ar[r, "2"] \& \ZZ\fgen{b}
        \end{tikzcd}
      \Big) \otimes \Big(
        \begin{tikzcd}[ampersand replacement=\&,
                       cramped, column sep=scriptsize]
          \ZZ\fgen{c} \ar[r, "4"] \& \ZZ\fgen{d}
        \end{tikzcd}
      \Big),
  \intertext{where we add generators to make it more clear. Multiplying,}
    &\iso \begin{tikzcd}[ampersand replacement=\&,
                         cramped, column sep=scriptsize]
      \ZZ\fgen{a \otimes c} \ar[r, "\partial_2"] \&
        \ZZ\fgen{b \otimes c, a \otimes d} \ar[r, "\partial_1"] \&
        \ZZ\fgen{b \otimes d}
    \end{tikzcd}
  \end{align*}
  where
  \begin{align*}
    \partial(a \otimes c) &=        (\partial a \otimes c)
                             + (-1) (a \otimes \partial c)
                           = (2b \otimes c) - (a \otimes 4d) \\
    \partial(b \otimes c) &=   (\partial b \otimes c)
                             + (b \otimes \partial c)
                           = 4 (b \otimes d) \\
    \partial(a \otimes d) &=        (\partial a \otimes d)
                             + (-1) (a \otimes \partial d)
                           = 2 (b \otimes d).
  \end{align*}
  We can also confirm
  \[
    \partial \partial (a \otimes c)
      = \partial (2(b \otimes c) - 4(a \otimes d))
      = 2 \cdot 4(b \otimes d) - 4 \cdot 2(b \otimes d)
      = 0.
  \]
  We can then compute \(H_0\) of this complex to be
  \(\frac{\ZZ\fgen{b \otimes d}}{\img \partial_1} = \ZZ/2\ZZ\),
  and \(H_1\) to be
  \(
    \ker \partial_1 / \img \partial_2
      = \frac{\ZZ\fgen{ b \otimes c - 2(a \otimes d)}}
             {\ZZ\fgen{2b \otimes c - 4(a \otimes d)}}
      \iso \ZZ/2\ZZ
  \).
  Moreover, we can notice \(\partial_2\) is injective,
  so all higher order homology groups are zero, as expected.
\end{example}

\begin{example}
  To compute \(\Tor_i(\ZZ/2\ZZ, \ZZ/3\ZZ)\), we have
  \begin{align*}
    \ZZ/2\ZZ \otimes^{\mathbb L} \ZZ/3\ZZ
      &\iso \Big(
        \begin{tikzcd}[ampersand replacement=\&,
                       cramped, column sep=scriptsize]
          \ZZ \ar[r, "2"] \& \ZZ
        \end{tikzcd}
      \Big) \otimes \ZZ/3\ZZ \\
      &\iso \begin{tikzcd}[ampersand replacement=\&,
                           cramped, column sep=scriptsize]
          \ZZ/3\ZZ \ar[r, "2"] \& \ZZ/3\ZZ
        \end{tikzcd}
  \end{align*}
  which means
  \begin{align*}
    \Tor_0(\ZZ/2\ZZ, \ZZ/3\ZZ) &\iso 0 \\
    \Tor_1(\ZZ/2\ZZ, \ZZ/3\ZZ) &\iso 0.
  \end{align*}
\end{example}

\begin{example}
  For \(\Tor_i(\ZZ/3\ZZ, \ZZ)\), note that
  \[
    \ZZ/3\ZZ \otimes^{\mathbb L} \ZZ \iso \ZZ/3\ZZ,
  \]
  where we interpret \(\ZZ/3\ZZ\) as
  \[
    \begin{tikzcd}
      \cdots \ar[r] &
        0 \ar[r] &
        \ZZ/3\ZZ \ar[r] &
        0 \ar[r] &
        \cdots
    \end{tikzcd}
  \]
  so
  \[
    \Tor_0(\ZZ/3\ZZ, \ZZ) \iso \ZZ/3\ZZ
  \]
  and \(\Tor_i = 0\) for all other \(i\).
\end{example}

\begin{example}
  If \(R\) is a field, and \(V\) and \(W\) are vector spaces over \(R\),
  then \(\Tor_0(V, W) \iso V \otimes_R W\) and \(\Tor_i(V, W) \iso 0\)
  for all other \(i\).
\end{example}

We will now prove that the \(\Tor\) groups are well defined
without reference to the derived categories.
\begin{proof}[\(\Tor\) groups are well defined]
  Suppose \(M\) and \(N\) are two \(R\)-modules, and suppose
  \(F_*\) and \(F_*'\) are two free resolutions of \(M\).
  We will prove that \(F_* \otimes_R N\) and \(F_*' \otimes_R N\)
  have the same homology groups.
  In fact, we will show that they are chain homotopy equivalent.

  By the fundamental theorem of homological algebra,
  there exists a unique (up to chain homotopy) map
  \(f \colon F_* \to F_*'\) lifting \(1_M\) and
  \(g \colon F_*' \to F_*\) lifting \(1_M\).

  Note that \(f \circ g\) is a map from \(F_*' \to F_*'\) lifting \(1_M\),
  so by the fundamental theorem,
  \(f \circ g\) must be chain homotopic to \(1_{F_*'}\).
  It follows that
  \((f \circ g) \otimes_R N = (f \otimes N) \circ (g \otimes N)\),
  which is chain homotopic to \(1_{F_*' \otimes N}\).

  Similarly, \((g \circ f) \otimes_R N\) is chain homotopic to
  \(1_{F_* \otimes N}\).
  Therefore, \(g \otimes_R N\) and \(f \otimes_R N\) are inverses
  up to chain homotopy.
  This proves that \(F_* \otimes_R N\) and \(F_*' \otimes_R N\)
  are chain homotopy equivalent.
\end{proof}

\subsection{On internal \texorpdfstring{\(\Hom\)}{Hom}s}
\begin{definition}
  If \(F_*\) is a chain complex of free \(R\)-modules and
  \(N\) is an \(R\)-module, considered as a chain concentrated in degree \(0\).
  Then we may form a new chain complex \(\ul\Hom_{\der(R)}(F_*, N)\)
  with
  \[
    \ul\Hom(F_*, N)_n = \ul\Hom_{\cRmod}(F_{-n}, N).
  \]
\end{definition}

\begin{example}
  If \(R = \ZZ\),
  \[
    F_* = \bigg(\begin{tikzcd}[row sep=-1ex]
      \cdots \ar[r] &
        0 \ar[r] &
        \ZZ \ar[r, "4"] &
        \ZZ \ar[r] &
        0 \ar[r] &
        \cdots \\
      & & \deg 1 & \deg 0
    \end{tikzcd}
    \bigg)
  \]
  and \(N = \ZZ/2\ZZ\).
  Then
  \[
    \ul\Hom_{\der(\ZZ)}(F_*, N) \iso
    \bigg(
      \begin{tikzcd}[row sep=-1ex]
        \cdots \ar[r] &
          0 \ar[r] &
          \ZZ/2\ZZ \ar[r, "\partial"] &
          \ZZ/2\ZZ \ar[r] &
          0 \ar[r] &
          \cdots \\
        & & \deg 0 & \deg {-1}
      \end{tikzcd}
    \bigg)
  \]
  The degree \(0\) \(R\)-module is
  \(\ul\Hom(F_0, N) \iso \ul\Hom(\ZZ, \ZZ/2\ZZ) \iso \ZZ/2\ZZ\),
  and the degree \(1\) \(R\)-module is \(\ul\Hom(F_1, N) \iso \ZZ/2\ZZ\).

  We can see that the differential \(\partial\) exists because
  given a map in \(\ul\Hom(F_0, N)\),
  we can compose it with the \(4\) map to get a map in \(\ul\Hom(F_1, N)\).
\end{example}

\begin{definition}
  If \(M\) and \(N\) are \(R\)-modules, then \(\ul\Hom_{\der(R)}(M, N)\)
  can be computed by replacing \(M\) with a free resolution and
  then applying the above construction.
\end{definition}

\begin{definition}
  If \(M\) and \(N\) are \(R\)-modules, then the \vocab{\(\Ext\) functor} is
  \begin{align*}
    \Ext_R^i(M, N) &= H_{-i}(\ul\Hom_{\der(R)}(M, N)) \\
      &= H_{-i}(\ul\Hom_{\der(R)}(F_*, N)),
  \end{align*}
  where \(F_*\) is any free resolution of \(M\).
\end{definition}

We will use \(\Tor\) and \(\Ext\) to compute homology with coefficients from
homology with \(\ZZ\) coefficients.






\end{document}
