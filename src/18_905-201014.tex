\documentclass{standalone}
\usepackage{chez}

\begin{document}
\chapter{October 14, 2020}

\section{Some algebra}
If \(A\) and \(B\) are abelian groups,
and \(f, g \colon A \to B \in \Hom_{\cAb}(A, B)\),
we can compute \(f + g\) and \(f - g\).
Therefore, we can make the set \(\Hom_{\cAb}(A, B)\) into an abelian group,
denoted \(\ul{\Hom}_\cAb(A, B)\).

More generally, if \(R\) is a commutative ring and \(\cRmod\) is
the category of \(R\)-modules, then \(\Hom_\cRmod(M, N)\) can be
given the structure of an \(R\)-module.
We denote this \(R\)-module \(\underline{\Hom}_\cRmod(M, N)\).

\begin{example}
  Linear maps between two vector spaces over a field form a vector space.
\end{example}

This construction \((M, N) \mapsto \ul\Hom_{\cRmod}(M, N)\) is functorial,
i.e.\ it is natural in its inputs. More specifically,
\begin{enumerate}
  \item If \(N \to N'\) is a map of \(R\)-modules, then any map \(M \to N\)
        gives a composite map \(M \to N \to N'\). There is a \(R\)-module map
        \(\ul\Hom_\cRmod(M, N) \to \ul\Hom_\cRmod(M, N')\).
  \item If \(M \to M'\) is an \(R\)-module map, then there is a corresponding
        map, for every \(N\),
        \(\ul\Hom_\cRmod(M', N) \to \ul\Hom_\cRmod(M, N)\).
\end{enumerate}
Note that in the first map, it goes from \(N\) to \(N'\),
and in the second map, it goes from \(M'\) to \(M\).
We can summarize this by saying that there is a functor
\begin{align*}
  \ul\Hom_\cRmod \colon& (\cRmod)^\op \times (\cRmod) \to \cRmod \\[-1ex]
    & (M, N) \mapsto \ul\Hom_\cRmod(M, N)
\end{align*}

Some categories \(\mathcal C\), like \(\cRmod\), have \vocab{internal Hom}s,
which are functors
\[
  \ul\Hom_{\mathcal C} \colon \mathcal C^\op \times \mathcal C \to \mathcal C.
\]
Here, \(\mathcal C^\op \times \mathcal C\) refers to the product category
of \(\mathcal C^\op\) and \(\mathcal C\) in \(\cCat\).

For example, the category \(\cSet\) has an internal \(\Hom\) with
\[
  \ul\Hom_\cSet(A, B) = \Hom_\cSet(A, B).
\]

If \(A, B, C \in \cSet\) there is a \vocab{currying isomorphism}
\[
  \Hom_\cSet(A \times B, C) \iso \Hom_\cSet(A, \Hom_\cSet(B, C))
\]
where a function \(f \colon A \to \Hom_\cSet(B, C)\) is sent to the function
\(g \colon A \times B \to C\) given by
\[
  g(a, b) = (f(a))(b).
\]
Whenever there is an internal \(\Hom\), there is generally a isomorphism
of this flavor.
The analog of the currying isomorphism in \(\cRmod\) is the following:
\begin{theorem}
  Let \(R\) be a commutative ring.
  There is a functor \vocab{tensor product}
  \[
    \otimes_R \colon \cRmod \times \cRmod \to \cRmod
  \]
  such that
  \[
    \ul\Hom_\cRmod(A \otimes_R B, C)
      \iso \ul\Hom_\cRmod(A, \ul\Hom_\cRmod(B, C))
  \]
  for all \(R\)-modules \(A, B, C\).
  Note that it is customary to write
  \(A \otimes_R B \coloneqq \otimes_R(A, B)\).
  Furthermore, the isomorphism is natural in \(A\), \(B\), and \(C\),
  and this uniquely determines the \(\otimes_R\) functor.
\end{theorem}

\begin{remark}
  The functors \(\otimes_R\) and \(\ul\Hom_\cRmod\) are \vocab{adjoint}.
  This means that they determine one another.
\end{remark}

\begin{proof*}{Sketch}
  First we will define the functor
  \[
    \otimes_R \colon \cRmod \times \cRmod \to \cRmod
  \]
  and then check that the isomorphism holds.
  \begin{definition}
    If \(A, B \in \cRmod\), then \(A \otimes_R B\) is the \(R\)-module
    \begin{itemize}[nosep]
      \item generated by the symbols \(a \otimes b\) where
            \(a \in A\) and \(b \in B\),
      \item with the relations
            \begin{gather*}
              a \otimes (b + b') = a \otimes b + a \otimes b' \\
              (a + a') \otimes b = a \otimes b + a' \otimes b \\
              (r a) \otimes b = a \otimes (r b) = r (a \otimes b)
            \end{gather*}
            for all \(a, a' \in A\), \(b, b' \in B\), and \(r \in R\).
    \end{itemize}
  \end{definition}
  
  We want to show that a map of \(R\)-modules \(A \otimes_R B \to C\)
  is determined by where it sends the generators \(a \otimes b\).
  Given a map \(f \colon A \otimes_R B \to C\), we can determine
  a function \(g \colon A \to \ul\Hom_\cRmod(B, C)\) by
  \[
    (g(a))(b) = f(a \otimes b).
  \]
  Conversely, given \(g\), we define
  \[
    f(a \otimes b) = (g(a))(b).
  \]
  The relations on \(A \otimes_R B\) are designed to ensure that \(g\) is
  an \(R\)-module map if and only if \(f\) is an \(R\)-module map.
\end{proof*}

Note that \(A \otimes_R B \iso B \otimes_R A\).

\subsection{Tensor product examples}
Suppose \(R = \ZZ\). Let's calculate \(A \otimes_\ZZ B\) for various
\(Z\)-modules \(A\) and \(B\).

\begin{example}
  The \(\ZZ\)-module \(\ZZ/2\ZZ \otimes_\ZZ \ZZ/4\ZZ\) is an abelian group
  with \(8\) generators
  \[
    0 \otimes 0, \qquad
    0 \otimes 1, \qquad
    0 \otimes 2, \qquad
    0 \otimes 3, \qquad
    1 \otimes 0, \qquad
    2 \otimes 1, \qquad
    3 \otimes 2, \qquad
    4 \otimes 3.
  \]
  We have the relation
  \[
    0 \otimes 2 = (0 \cdot 0) \otimes 2 = 0 (0 \otimes 2) = 0.
  \]
  Similarly, \(0 \otimes 0\), \(0 \otimes 1\), \(0 \otimes 3\), and
  \(1 \otimes 0\) are all trivial.
  Therefore, \(\ZZ/2\ZZ \otimes_\ZZ \ZZ/4\ZZ\) is generated by
  \[
    1 \otimes 1, \qquad
    1 \otimes 2, \qquad
    1 \otimes 3.
  \]
  Note that we have
  \begin{align*}
    1 \otimes 1 + 1 \otimes 1 + 1 \otimes 1 + 1 \otimes 1 &= 1 \otimes 4 = 0 \\
    1 \otimes 1 + 1 \otimes 1 + 1 \otimes 1 &= 1 \otimes 3 \\
    1 \otimes 1 + 1 \otimes 1 &= 1 \otimes 2.
  \end{align*}
  Therefore, the whole \(R\)-module is generated by \(1 \otimes 1\).
  Moreover, note
  \[
    1 \otimes 1 + 1 \otimes 1 = (1 + 1) \otimes 1 = 0 \otimes 1 = 0.
  \]
  Therefore, \(1 \otimes 1\) has order \(2\) and
  \[
    \ZZ/2\ZZ \otimes_\ZZ \ZZ/4\ZZ \iso \ZZ/2\ZZ.
  \]
\end{example}

Let's check an example of the theorem. It states
\[
  \Hom_\cAb(\ZZ/4\ZZ \otimes_\ZZ \ZZ/2\ZZ, \ZZ/2\ZZ)
    \iso \Hom_\cAb(\ZZ/2\ZZ, \ZZ/2\ZZ)
    \iso \Hom_\cAb(\ZZ/4\ZZ, \ul\Hom_\cAb(\ZZ/2\ZZ, \ZZ/2\ZZ)).
\]
Consider \(\Hom_\cAb(\ZZ/2\ZZ, \ZZ/2\ZZ)\).
The elements are the zero map and the identity map.
Therefore, the theorem claims
\[
  \Hom_\cAb(\ZZ/2\ZZ, \ZZ/2\ZZ) \iso \Hom_\cAb(\ZZ/4\ZZ, \ZZ/2\ZZ).
\]
This is indeed true because \(\Hom_\cAb(\ZZ/4\ZZ, \ZZ/2\ZZ)\) has two elements
that are determined where the morphism sends the generator \(1\), for which
there are two choices.







\end{document}
