\documentclass{standalone}
\usepackage{chez}

\begin{document}
\chapter{October 19, 2020}

If \(C_*\) and \(D_*\) are chain complexes, then \(\Hom(C_*, D_*)\)
is not obviously a chain complex.


\section{Some algebra}
Suppose
\[
  \begin{tikzcd}
  	0 \ar[r] &
  	A \ar[r] &
  	B \ar[r] &
  	C \ar[r] &
  	0
  \end{tikzcd}
\]
is a short exact sequence of \(R\)-modules and \(M\) is another \(R\)-module.
Then, note that the sequence
\[
  \begin{tikzcd}
  	0 \ar[r] &
  	A \otimes_R M \ar[r] &
  	B \otimes_R M \ar[r] &
  	C \otimes_R M \ar[r] &
  	0
  \end{tikzcd}
\]
is in general not exact.
\begin{example}
  Let \(R = \ZZ\) and consider the short exact sequence
  \[
    \begin{tikzcd}
    	0 \ar[r] &
    	\ZZ \ar[r, "2"] &
    	\ZZ \ar[r] &
    	\ZZ/2\ZZ \ar[r] &
    	0
    \end{tikzcd}
  \]
  Tensoring with \(\ZZ/2\ZZ\) yields
  \[
    \begin{tikzcd}[row sep=0.1ex]
    	0 \ar[r] &
        \ZZ/2\ZZ \otimes_\ZZ \ZZ \ar[r] &
        \ZZ/2\ZZ \otimes_\ZZ \ZZ \ar[r] &
        \ZZ/2\ZZ \otimes_\ZZ \ZZ/2\ZZ \ar[r] &
        0 \\
    	0 \ar[r] &
        \ZZ/2\ZZ \ar[r] &
        \ZZ/2\ZZ \ar[r] &
        \ZZ/2\ZZ \ar[r] &
        0
    \end{tikzcd}
  \]
  However, the multiplication by \(2\) map becomes
  the zero map when we tensor by \(\ZZ/2\ZZ\),
  so the first \(\ZZ/2\ZZ \to \ZZ/2\ZZ\) map becomes \(0\).
  However, the zero map is not injective,
  so the sequence we get is not exact.
\end{example}

\begin{proposition}<module-exactness-tensor>
  If \(M\) is an \(R\)-module, then the functor
  \[
    {-} \otimes_R M \colon \cRmod \to \cRmod
  \]
  is \vocab{right exact}.
  In other words, if
  \[
    \begin{tikzcd}
    	A \ar[r] &
    	B \ar[r] &
    	C \ar[r] &
    	0
    \end{tikzcd}
  \]
  is an exact sequence of \(R\)-modules, then so is
  \[
    \begin{tikzcd}
    	A \otimes_R M \ar[r] &
    	B \otimes_R M \ar[r] &
    	C \otimes_R M \ar[r] &
    	0
    \end{tikzcd}
  \]
\end{proposition}
\begin{corollary}
  In fact,
  \[
    \begin{tikzcd}
    	A \ar[r] &
    	B \ar[r] &
    	C \ar[r] &
    	0
    \end{tikzcd}
  \]
  is exact if and only if
  \[
    \begin{tikzcd}
    	A \otimes_R M \ar[r] &
    	B \otimes_R M \ar[r] &
    	C \otimes_R M \ar[r] &
    	0
    \end{tikzcd}
  \]
  is exact for every \(R\)-module \(M\).
\end{corollary}
\begin{proof*}{Sketch}
  Take \(M = R\).
\end{proof*}

To prove something about tensor products,
it is often helpful to think about
the tensor products as internal \(\Hom\)s.
In particular, we have the related theorem
\begin{proposition}<module-exactness-hom>
  A sequence of \(R\)-modules
  \[
    \begin{tikzcd}
    	A \ar[r] &
    	B \ar[r] &
    	C \ar[r] &
    	0
    \end{tikzcd}
  \]
  is exact if and only if for all \(R\)-modules \(N\),
  \[
    \begin{tikzcd}
    	0 \ar[r] &
    	\ul\Hom_{\cRmod}(C, N) \ar[r] &
    	\ul\Hom_{\cRmod}(B, N) \ar[r] &
    	\ul\Hom_{\cRmod}(A, N)
    \end{tikzcd}
  \]
  is exact.
\end{proposition}

\begin{proof}[\cref{prop:module-exactness-hom} \(\implies\)
  \cref{prop:module-exactness-tensor}]
  Suppose \(A \to B \to C \to 0\) is exact.
  To prove
  \[
    \begin{tikzcd}
    	A \otimes_R M \ar[r] &
    	B \otimes_R M \ar[r] &
    	C \otimes_R M \ar[r] &
    	0
    \end{tikzcd}
  \]
  is exact, by \cref{prop:module-exactness-hom} it suffices to check that
  \[
    \begin{tikzcd}
    	0 \ar[r] &
    	\ul\Hom(C \otimes_R M, N) \ar[r] &
    	\ul\Hom(B \otimes_R M, N) \ar[r] &
    	\ul\Hom(A \otimes_R M, N)
    \end{tikzcd}
  \]
  is exact for every \(N\).
  This is equivalent to checking that
  \[
    \begin{tikzcd}
    	0 \ar[r] &
    	\ul\Hom(C, \ul\Hom(M, N)) \ar[r] &
    	\ul\Hom(B, \ul\Hom(M, N)) \ar[r] &
    	\ul\Hom(A, \ul\Hom(M, N))
    \end{tikzcd}
  \]
  is exact for every \(N\).
  By \cref{prop:module-exactness-hom} again, this is true.
\end{proof}

\begin{proof}[\cref{prop:module-exactness-hom}]
  Suppose
  \(
    \begin{tikzcd}[cramped, column sep=small]
      A \ar[r, "\varphi"] & B \ar[r, "\psi"] & C \ar[r] & 0
    \end{tikzcd}
  \)
  is exact. We want to show
  \[
    \begin{tikzcd}
    	0 \ar[r] &
    	\ul\Hom(C, N) \ar[r] &
    	\ul\Hom(B, N) \ar[r] &
    	\ul\Hom(A, N)
    \end{tikzcd}
  \]
  is exact.

  Exactness at \(\ul\Hom(C, N)\) is equivalent to the map
  \(\ul\Hom(C, N) \to \ul\Hom(B, N)\) being injective.
  This is indeed true, because \(\psi\) is surjective,
  so the composite \(B \to C \to N\) determines a map \(C \to N\).

  Consider the diagram
  \[
    \begin{tikzcd}
    	A \ar[r, "\varphi"] \ar[rd, "0"'] &
        B \ar[r, "\psi"] \ar[d, "f" pos=0.3] &
        C \ar[r] \ar[ld, "g"]&
        0 \\
    	& N & &
    \end{tikzcd}
  \]
  Exactness at \(\ul\Hom(B, N)\) is equivalent to saying that a
  map \(f \colon B \to N\) is restricted to the zero map \(A \to B \to N\)
  if and only if \(f\) can be written as a composite \(B \to C \to N\).
  The backward direction is clear because any composite \(A \to B \to C \to N\)
  is zero because \(A \to B \to C\) is exact.
  We want to find a function \(g \colon C \to N\)
  such that \(f = g \circ \psi\).
  If \(f \circ \varphi = 0\), then note that \(C \iso B / \img \varphi\).
  We can then assign \(g [b] = f b\),
  where \([b]\) denotes the equivalent class representing \(b\).
  This gives \(f = g \circ \psi\), as desired.
\end{proof}

We have now proved that if
\[
  \begin{tikzcd}
  	0 \ar[r] &
  	A \ar[r] &
  	B \ar[r] &
  	C \ar[r] &
  	0
  \end{tikzcd}
\]
is exact and \(M\) is an \(R\)-module, then
\[
  \begin{tikzcd}
  	A \otimes_R M \ar[r] &
  	B \otimes_R M \ar[r] &
  	C \otimes_R M \ar[r] &
  	0
  \end{tikzcd}
\]
is exact, but
\[
  \begin{tikzcd}
  	0 \ar[r] &
  	A \otimes_R M \ar[r] &
  	B \otimes_R M \ar[r] &
  	C \otimes_R M \ar[r] &
  	0
  \end{tikzcd}
\]
is not necessarily exact.

However, if we let \(M = R\), then it is exact.
Furthermore, if we take \(M = R \oplus R\), the sequence will stay exact.
In particular, it will become
\[
  \begin{tikzcd}
    0 \arrow[r] &
    A \oplus A \arrow[r] &
    B \oplus B \arrow[r] &
    C \oplus C \arrow[r] &
    0
  \end{tikzcd}
\]
In general, if \(M\) is a free \(R\)-module, then tensoring with \(M\)
gives an exact sequence and not just a right exact sequence.

\begin{note}
  If \(X\) is a topological space, then \(\Ccell_*(X; R)\) is
  a chain complex of free \(R\)-modules.
\end{note}







\end{document}
