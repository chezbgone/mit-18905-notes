\documentclass{standalone}
\usepackage{chez}

\begin{document}
\chapter{September 21, 2020}

The final piece that we need before we do calculations is
the following theorem.
\begin{theorem}[Excision]
  Let \((X, A)\) be a pair of spaces, and assume there is a subspace
  \(B \subseteq X\) such that
  \begin{enumerate}[nosep]
    \item \(\ol A \subseteq \operatorname{int} B\) and
    \item \(A \to B\) is a deformation retract.
  \end{enumerate}
  Then for all \(m\) there is an isomorphism
  \[
    H_m(X, A) \iso H_m(X/A, *)
      \iso \begin{cases}
        H_m(X/A) & m > 0 \\[-1ex]
        \ker(\ZZ \pi_0 X \to \ZZ) & m = 0.
      \end{cases}
  \]
\end{theorem}
The condition that there is a subspace \(B\) is not particularly restrictive,
as it almost always exists. Once this condition is satisfied, this theorem says
that relative homology is the same thing as the absolute homology of
a quotient, as seen from the snake lemma.

\begin{definition}
  Suppose \(U \subseteq A \subseteq X\). Such a triple is said to be
  \vocab{excisive} if \(\ol{U} \subseteq \Int A\).
  The inclusion \((X - U, A - U) \subseteq X, A\)
  is called an \vocab{excision}.
\end{definition}
\begin{theorem}<excision>
  An excision induces an isomorphism
  \[
    H_m(X - U, A - U) \iso H_m(X, A).
  \]
\end{theorem}

Before we prove this result, we will first make a definition and summarize
what we will have afterwards.

\begin{definition}
  Two maps of pairs \(f, g \colon (X, A) \to (Y, B)\) are \vocab{homotopic}
  if there exists a \(h \colon X \times [0, 1] \to Y\) such that
  \begin{enumerate}[nosep]
    \item \(h(x, 0) = f\),
    \item \(h(x, 1) = g\),
    \item \(h(a, t) \in B\) for all \(a \in A\) and \(t \in [0, 1]\).
  \end{enumerate}
\end{definition}
This allows us to define \(\cat{Ho}(\cTop_2)\) analogous to how we defined
\(\cHoTop\).

\subsection{Summary}
We have:
\begin{itemize}[nosep]
  \item A sequence of functors \(H_n \colon \cTop_2 \to \cAb\)
  for all \(n \in \ZZ\).
  \item A sequence of natural transformations
  \(\partial \colon H_n(X, A) \to H_{n-1}(A) \coloneqq H_{n-1}(A, \nullset)\)
  from the snake lemma.
\end{itemize}
such that:
\begin{enumerate}[nosep]
  \item For any pair \((X, A)\), the sequence
  \[
    \begin{tikzcd}
      \cdots \ar[r] &
      H_{q+1}(X, A) \ar[r, "\partial"] &
      H_q(A) \ar[r] &
      H_q(X) \ar[r] &
      H_q(X, A) \ar[r, "\partial"] &
      \cdots
    \end{tikzcd}
  \]
  is exact.

  \item If \(f_0, f_1 \colon (X, A) \to (Y, B)\) are homotopic,
  then \(H_n(f_0) = H_n(f_1)\).

  \item Excisions induce homology isomorphisms.

  \item\label{axiom-4} (Dimension axiom) The groups
  \[
    H_n(*) \iso \begin{cases*}
      \ZZ & if \(n = 0\) \\[-1ex]
      0   & otherwise.
    \end{cases*}
  \]

  \item Let \(X_i\) be a collection of spaces
  indexed by \(I\). Then
  \[
    H_m\parens*{\coprod_{i \in I} X_i} \iso \bigoplus_{i \in I} H_m(X_i).
  \]
\end{enumerate}

\begin{theorem}[Eilenberg-Steenrod]<eilenberg-steenrod>
  These facts characterize the \(H_m\) functors.
  In other words, any other functor \(\cTop \to \cAb\)
  satisfying these properties would be isomorphic to \(H_m\).
\end{theorem}

\begin{remark}
   A set of functors \(E_n \colon \cTop_2 \to \cAb\) satisfying all the axioms
   except for~\ref{axiom-4} is called an \vocab{extraordinary homology theory}.
\end{remark}

If we modify the dimension axiom so that the homology of a point some data
that is not the integers, we can still compute things.
\begin{question}
  Can one define such things that are computable but contain more refined than that seen by \(H_m\)?
\end{question}
This is an active area of research, namely asking which data are useful.
Examples include
  K-theory,
  bordism theory,
  topological modular fans,
  topological automorphic forms, and
  Morava E-theories.

\section{Calculations}
Rather than proving excision right now, we can start to do some calculations.

Let's calculate \(H_m(S^1)\) for various \(m \in \ZZ\).
Consider \(S^1 \iso [0, 1]/\set{0, 1}\). The long exact sequence associated to
the pair \(([0, 1], \set{0, 1})\) is
\[
  \begin{tikzcd}
    \cdots \ar[r] &
    H_2(\set{0, 1}) \ar[r] &
    H_2([0, 1]) \ar[r] &
    H_2([0, 1], \set{0, 1})
      \ar[dll, "\delta"' pos=0.9, out=-30, in=160, looseness=0.7] & \\
  & H_1(\set{0, 1}) \ar[r] &
    H_1([0, 1]) \ar[r] &
    H_1([0, 1], \set{0, 1})
      \ar[dll, "\delta"' pos=0.9, out=-30, in=160, looseness=0.7] & \\
  & H_0(\set{0, 1}) \ar[r] &
    H_0([0, 1]) \ar[r] &
    H_0([0, 1], \set{0, 1}) \ar[r] & \cdots
  \end{tikzcd}
\]

We can start computing some of the elements in this chain.
\begin{itemize}[nosep]
  \item \(H_2(\set{0, 1}) \iso H_2(\set{0}) \sqcup H_2(\set{1}) \iso 0\)
  \item \(H_2([0, 1]) \iso 0\) because the interval is contractable.
  \item This means that the map \(H_2(\set{0, 1}) \to H_2([0, 1])\) is \(0\).
  \item Similarly, we have \(H_1(\set{0, 1}) \iso H_2([0, 1]) \iso 0\).
  \item By exactness, we have that the maps in the chain
    \(H_2([0, 1]) \to H_2([0, 1], \set{0, 1}) \to H_1(\set{0, 1})\)
    are the zero maps.
  \item From the dimension axiom and additivity,
    \(H_0(\set{0, 1}) \iso \ZZ \oplus \ZZ\).
  \item \(H_0([0, 1]) \iso \ZZ\) because \([0, 1]\) is contractable.
\end{itemize}
This gives us the updated diagram
\[
  \begin{tikzcd}
    \cdots \ar[r] &
    0 \ar[r, "0"] &
    0 \ar[r, "0"] &
    0 \ar[dll, "\delta = 0"' pos=0.9, out=-30, in=160, looseness=0.7] \\
  & 0 \ar[r, "0"] &
    0 \ar[r] &
    H_1([0, 1], \set{0, 1})
      \ar[dll, "\delta"' pos=0.9, out=-30, in=160, looseness=0.7] \\
  & \ZZ \oplus \ZZ \ar[r, "f"] &
    \ZZ \ar[r, "g"] &
    H_0([0, 1], \set{0, 1}) \ar[r] & 0 \ar[r] &\cdots
  \end{tikzcd}
\]
Since \(H_0\) measures path components, the map
\(f \colon \ZZ \oplus \ZZ \to \ZZ\) maps
\((1, 0) \mapsto 1\) and \((0, 1) \mapsto 1\). In particular, it maps the two
elements representing the path components of \(\set{0, 1}\) to the single
path component of \([0, 1]\). By exactness and the first isomorphism theorem,
\(H_1([0, 1], \set{0, 1})\) is isomorphic to \(\ker f = \ZZ\).
Similarly, we know that \(\img f = \ZZ = \ker g\) by exactness, so \(g\) is
the zero map. This means that \(H_0([0, 1], \set{0, 1}) \iso 0\).

Overall, we have deduced from this calculation that
\[
  H_0([0, 1], \set{0, 1}) \iso 0 \qquad\qquad
  H_1([0, 1], \set{0, 1}) \iso \ZZ \qquad\qquad
  H_2([0, 1], \set{0, 1}) \iso 0.
\]
By excision, we have for \(m > 0\)
\[
  H_m([0, 1], \set{0, 1}) = H_m(S^1, *).
\]
This gives
\[
  H_0(S^1) = \ZZ \oplus \tilde H_0(S^1) \iso \ZZ \qquad\qquad
  H_1(S^1) = \ZZ \qquad\qquad
  H_2(S^1) = 0.
\]
What this means geometrically is that \(S^1\) has no \(2\)-dimensional holes,
one \(1\)-dimensional hole, and one path component.

If we extend the exact sequence to the left, we can continue to calculate
\[
  H_m(S^1) \iso \begin{cases*}
    \ZZ & if \(m = 0, 1\) \\[-1ex]
    0   & otherwise.
  \end{cases*}
\]
As a more general fact, we have
\[
  H_m(S^q) \iso \begin{cases*}
    \ZZ & if \(m = 0, q\) \\[-1ex]
    0   & otherwise.
  \end{cases*}
\]


\begin{corollary}
  If \(q \neq r\), \(S^q\) and \(S^r\) are not homotopy equivalent.
\end{corollary}
\begin{proof*}{Sketch}
  They would have the same homology groups otherwise
\end{proof*}
\begin{corollary}
  If \(q \neq r\), \(\RR^q\) and \(\RR^r\) are not homeomorphic.
\end{corollary}
\begin{proof}
  Suppose they were. Then \(\RR^q - \set{0}\) would be homeomorphic to
  \(\RR^r - \set{0}\). However, there are deformation retracts
  \begin{gather*}
    S^{q-1} \hookrightarrow \RR^q - \set{0} \\[-1ex]
    S^{r-1} \hookrightarrow \RR^r - \set{0},
  \end{gather*}
  meaning that \(S^{q-1}\) and \(S^{r-1}\) are homotopy equivalent,
  which is a contradiction.
\end{proof}


\begin{theorem}[Brouwer fixed point theorem]
  Let \(n \geq 2\).
  If \(f \colon D^n \to D^n\) is continuous, then there is a point
  \(x \in D^n\) such that \(f(x) = x\).
\end{theorem}
\begin{proof}
  Suppose there were no fixed point. Then one can define the continuous map
  \(g \colon D^n \to S^{n-1}\) that maps \(x\) to the point where the ray
  connecting \(f(x)\) to \(x\) hits the boundary.
  
  If \(x \in S^{n-1} \subseteq D(n)\) is on the boundary, \(g(x) = x\).
  Then, we have the maps
  \[
    \begin{tikzcd}
      S^{n-1} \ar[r, hook] \ar[rr, bend right, "\id"'] &
      D^n \ar[r, "g"] &
      S^{n-1}
    \end{tikzcd}
  \]
  Applying the functor \(H_{n-1}\), we have
  \[
    \begin{tikzcd}
      \ZZ \ar[r] \ar[rr, bend right, "\id"'] &
      0 \ar[r] &
      \ZZ
    \end{tikzcd}
  \]
  which is a contradiction.
\end{proof}




\end{document}
