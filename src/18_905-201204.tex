\documentclass{standalone}
\usepackage{chez}

\begin{document}
\chapter{December 04, 2020}

Recall that if \(X\) is a topological space,
we defined the \v{C}ech cohomology of a closed subset \(K \subseteq X\) to be
\[
  \check H^*(K) = \coprod_{K \subseteq U} H^*(U) / \sim,
\]
where \(U\) ranges over the set of open neighborhoods of \(K\),
where we consider two classes to be the same if
one is an inclusion of the other.

If \(L \subseteq K\) is a pair of closed subsets of \(X\),
then we can form
\[
  \check H^*(K, L) \coloneqq \coprod_{\scriptsize
    \begin{tikzcd}[cramped, column sep=1ex, row sep=1ex]
      L\ar[r, symbol=\subseteq]\ar[d, symbol=\subseteq]
        & V \ar[d, symbol=\subseteq] \\
      K\ar[r, symbol=\subseteq] & U
    \end{tikzcd}
  } H^*(U, V) / \sim
\]
where \((U, V)\) ranges over all pairs \(Y \subseteq U\) such that
\(V\) is an open neighborhood of \(L\) and
\(U\) is an open neighborhood of \(K\).


The theorems that we expect to hold from standard cohomology indeed hold.
\begin{theorem}[LES]
  Let \((K, L)\) be a closed pair in \(X\).
  Then there is a natural long exact sequence
  \[
    \begin{tikzcd}
      \cdots \ar[r] &
        \check H^p(K, L) \ar[r] &
        \check H^p(K) \ar[r] &
        \check H^p(L) \ar[r, "\delta"] &
        \check H^{p+q}(K, L) \ar[r] &
        \cdots
    \end{tikzcd}
  \]
\end{theorem}

\begin{theorem}[Excision]
  Suppose \(A\) and \(B\) are compact subsets of a Hausdorff space \(X\).
  Then the inclusion
  \[
    (B, A \intersect B) \subseteq (A \union B, A)
  \]
  induces an isomorphism
  \[
    \check H^p(A \union B, A) \iso \check H^p(B, A \intersect B)
  \]
  for all \(p\).
\end{theorem}

Before we had a cap produce for a closed subspace \(K \subseteq X\)
\[
  \cap \colon \check H^p(K) \otimes_\ZZ H_n(X, X - K) \to H_{n-p}(X, X - K).
\]
We can extend this to a \vocab{fully relative cap product}.
\begin{definition}
  Let \(L \subseteq K\) be a pair of closed subspaces of \(X\).
  Then there is a map
  \[
    \cap \colon \check H^p(K, L) \otimes_\ZZ H_n(X, X - K)
            \to H_{n-p}(X, X - K).
  \]
\end{definition}
Then, this fully relative cap produce commutes with Mayer-Vietoris sequence.
\begin{theorem}
  Let \(A\) and \(B\) be compact subsets of a Hausdorff space \(X\).
  Let \(x_{A \union B} \in H_n(X, X - A \union B)\) be a homology class.
  This gives us canonical restrictions
  \[
    x_A \in H_n(X, X-A),
    x_B \in H_n(X, X-B), \text{ and }
    x_{A \intersect B} \in H_n(X, X - A \intersect B).
  \]
  Then there is a map of long exact sequences
  \[
    \begin{tikzcd}[column sep=tiny]
      \cdots \ar[r] &
        \check H^p(A \union B) \ar[r]
          \ar[d, "{-} \cap x_{A \union B}"] &
        \check H^p(A) \oplus H^p(B) \ar[r]
          \ar[d, "({-} \cap x_A) \oplus ({-} \cap x_B)"] &
        \check H^p(A \intersect B) \ar[r, "\delta"]
          \ar[d, "{-} \cap x_{A \intersect B}"] &
        \check H^{p+1}(A \union B) \ar[r]
          \ar[d, "{-} \cap x_{A \union B}"] &
        \cdots \\
      \cdots \ar[r] &
        H_{n-p}(X, X - A \union B) \ar[r] &
        H_{n-p}(X, X - A) \oplus H_{n-p}(X, X-B) \ar[r] &
        H_{n-p}(X, X - A \intersect B) \ar[r, "\delta"] &
        H_{n-p-1}(X, X - A \union B) \ar[r] &
        \cdots
    \end{tikzcd}
  \]
\end{theorem}

\section{Poincar\' e duality, again}
Let \(M\) be an \(n\)-dimensional manifold and let \(K\) be a compact subset.
Recall that
\[
  H_n(M, M - K) \to \Gamma(K; o_M) =
                    \set{
                      f \colon K \to o_M \mid
                      K \overset{f}{\to} o_M \overset{\pi}{\to} K = \id
                    }.
\]
A \(\ZZ\)-orientation along a closed subset \(K\) is
a section of \(o_M\) over \(K\) (i.e.\ an element of \(\Gamma(K; o_M)\))
that restricts to a generator of \(H_n(M, M - \set{x})\) for every \(x \in K\).
The corresponding class in \(H_n(M, M - K)\) is called the
\vocab{fundamental class along \(K\)}, denoted \([M]_K\).

If \(L \subseteq K\) is an inclusion of compact subsets of \(M\),
then the map
\[
  H_n(M, M - K) \to H_n(M, M - L)
\]
sends \([M]_K\) to a fundamental class \([M]_L\).
In particular, an inclusion of subsets gives a restriction of the orientation.

Also, we have a cap product
\[
  \cap \colon \check H^p(K, L) \otimes_\ZZ H_n(M, M - K) \to
              H_{n - p}(M - L, M - K).
\]

\begin{theorem}[Poincar\'e duality]
  Let \(M\) be an \(n\)-dimensional manifold and
    let \(L \subseteq K\) be a pair of compact subspaces.
  Assume we have a \(\ZZ\)-orientation along \(K\)
    with fundamental class \([M]_K\).
  Then the map
  \[
    {-} \cap [M]_K \colon \check H^p(K, L) \to H_{n - p}(M - L, M - K)
  \]
  is an isomorphism.
\end{theorem}

\begin{proof*}{Sketch}
  Read Miller's notes for more details.
  \begin{enumerate}[nosep]
    \item Prove the theorem for \(M = \RR^n\),
      where \(K\) and \(L\) are compact convex subsets.
    \item Prove for \(M = \RR^n\) where \(K\) and \(L\) are
      a finite union of compact convex subsets of \(\RR^n\).
    \item Prove for \(M = \RR^n\) where \(K\) and \(L\) are
      any compact subsets of \(\RR^n\).
    \item Prove for arbitrary \(M \) where \(K\) and \(L\) are
      finite unions of compact Euclidean subspace of \(M\).
    \item Prove for arbitrary \(M \) where \(K\) and \(L\) are
      arbitrary compact subspaces. \pog
  \end{enumerate}
\end{proof*}

\subsection{Applications}
\begin{theorem}
  Let \(M\) be an \(n\)-dimensional manifold and \(K\) be a compact subset.
  A \(\ZZ\)-orientation along \(K\) determines \([M]_K \in H_n(M, M - K)\)
  and capping with it gives an isomorphism
  \[
    \check H^{n - p}(K) \to H_p(M, M - K).
  \]
\end{theorem}
\begin{corollary}[Alexander duality]
  Suppose \(K\) is a compact subset of \(\RR^n\).
  The composite
  \[
    \check H^{n - p}(K) \overset{\iso}{\to}
      H_p(\RR^n, \RR^n - K) \overset{\partial}{\to}
      \tilde H_{p - 1}(\RR^n - K)
  \]
  is an isomorphism.
\end{corollary}

\begin{example}[Jordan curve theorem]
  Suppose \(n = 2\).
  Let \(K\) be a closed loop in \(\RR^2\) with \(\check H^1(K) \iso \ZZ\).
  (This will almost always be the case except for pathological embeddings
  of the circle into \(\RR^2\).)

  Then \(\tilde H_0(\RR^2 - K) \iso \check H^1(K) \iso \ZZ\),
  so \(H_0(\RR^2 - K) \iso \ZZ \oplus \ZZ\).

  In other words, \(\RR^2 - K\) has two path components.
\end{example}

\begin{adhoctheorem}{Warning}
  The analog of this in \(\RR^3\) is false.
  In particular, there are maps \(f \colon S^2 \to \RR^3\) where
  \(\RR^3 - \img f\) does not give two path components.
  The counterexample is called the Alexander horned sphere.
\end{adhoctheorem}



\end{document}
