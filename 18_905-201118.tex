\documentclass{standalone}
\usepackage{chez}

\begin{document}
\chapter{November 18, 2020}

\begin{proposition}[Problem 1 of PSET]<homology-cohomology-pairing>
  If \(X\) is a finite type CW complex, then the map,
  adjoint to the Kronecker pairing
  \[
    H^q(X; \FF_2)
      \to \ul\Hom_{\FF_2\text{-}\cat{mod}}(H_q(X; \FF_2), \FF_2) % chktex 35
  \]
  is an isomorphism.
\end{proposition}
This is an example of a perfect pairing.

\begin{definition}
  A \vocab{perfect pairing} of two finitely generated free \(R\)-modules
  \(V\) and \(W\) is an \(R\)-linear map
  \[
    V \otimes_R W \to R
  \]
  such that the adjoint map
  \[
    V \to \ul\Hom_R(W, R)
  \]
  is an isomorphism of \(R\) modules.
\end{definition}

\begin{definition}
  An \(n\)-dimensional \vocab{manifold} \(M\) is a Hausdorff topological space
  such that every point has an open neighborhood homeomorphic to \(\RR^n\).
\end{definition}

\begin{example}
  A \(2\)-dimensional manifold is called a \vocab{surface}.
  Examples include \(S^2\),
                   \(T^2\),
                   the Klein bottle \(K\),
                   \(\RP^2\), and
                   \(\RR^2\).
  
  Non-examples include \(S^2 \vee S^2\), because at the wedge point,
  there is no open neighborhood homeomorphic to \(\RR^2\).

  Examples of 3D manifolds include \(\RR^3\),
                                   \(S^3\),
                                   \(S^1 \times S^1 \times S^1\),
                                   \(\RP^3\), etc.

  Other manifolds include
  \begin{itemize}[nosep]
    \item smooth algebraic varieties over \(\RR\) or \(\CC\),
    \item configuration spaces in physics.
  \end{itemize}
\end{example}

\begin{fact}
  Any compact manifold is homotopy equivalent to a finite type CW complex.
\end{fact}


\begin{theorem}[Poincar\'e duality]
  Let \(M\) be a compact \(n\)-dimensional manifold.
  Then there exists a unique class \([M] \in H_n(M; \FF_2)\)
  called the \vocab{fundamental class} of the manifold, such that
  for all \(p, q \in \ZZ\) such that \(p + q = n\), the map
  \[
    H^p(M; \FF_2) \otimes_{\FF_2} H^q(M; \FF_2) \xrightarrow{\cup}
      H^n(M; \FF_2) \xrightarrow{\angles{{-}, [M]}} \FF_2
  \]
  is a perfect pairing.
\end{theorem}
In other words, \(H^p(M; \FF_2)\) is canonically isomorphic to
the \(F_2\)-linear dual of \(H^q(M; \FF_2)\).

\begin{example}
  Suppose \(M\) is a compact 3D manifold with
  \(H^0(M; \FF_2) = \FF_2 \oplus \FF_2\) and
  \(H^1(M; \FF_2) = \FF_2 \oplus \FF_2 \oplus \FF_2\).
  Then we can determine \emph{all} of
  the homology and cohomology groups of \(M\) with \(\FF_2\) coefficients.
  In particular,
  \begin{align*}
    H^2(M; \FF_2) &\iso \ul\Hom(H^1(M; \FF_2), \FF_2)
                   \iso \FF_2 \oplus \FF_2 \oplus \FF_2 \\
    H^3(M; \FF_2) &\iso \ul\Hom(H^0(M; \FF_2), \FF_2)
                   \iso \FF_2 \oplus \FF_2,
  \end{align*}
  and all other cohomology groups vanish.
  All other cohomology groups vanish, because e.g.\
  \[
    H^4(M; \FF_2) \iso H^{-1}(M; \FF_2)^\vee \iso 0^\vee \iso 0.
  \]

  From \cref{prop:homology-cohomology-pairing}, we have
  \begin{align*}
    H_0(M; \FF_2) &\iso \FF_2 \oplus \FF_2 \\
    H_1(M; \FF_2) &\iso \FF_2 \oplus \FF_2 \oplus \FF_2 \\
    H_2(M; \FF_2) &\iso \FF_2 \oplus \FF_2 \oplus \FF_2 \\
    H_3(M; \FF_2) &\iso \FF_2 \oplus \FF_2,
  \end{align*}
  and \(H_q(M; \FF_2) \iso 0\) for \(q > 3\).
\end{example}


\begin{example}
  Recall
  \begin{align*}
    H^*(T^2; \FF_2) &= \FF_2[x, y, z]/(xy - z, x^2, y^2, xz, yz, z^2) \\
      &\iso \underbracket{\FF_2\fgen{1}}_{\deg 0} \oplus
            \underbracket{\FF_2\fgen{x, y}}_{\deg 1} \oplus
            \underbracket{\FF_2\fgen{z}}_{\deg 2}.
  \end{align*}
  The fundamental class of the torus is
  an element of \(H_2(T^2; \FF_2) \iso \FF_2\fgen{\delta_z}\).
  The fundamental class is \(\delta_z\).

  By Poincar\'e duality, there is a perfect pairing
  \[
    P \colon H^1(T^2; \FF_2) \otimes_{\FF_2} H^1(T^2; \FF_2) \to \FF_2
  \]
  where
  \begin{align*}
    P(x \otimes y) &= \angles{xy, \delta_z}
                    = \angles{z, \delta_z}
                    = \delta_z(z) = 1 \\
    P(x \otimes x) &= \angles{xx, \delta_z}
                    = \angles{0, \delta_z}
                    = \delta_z(0) = 0,
  \end{align*}
  etc.

  There is another pairing
  \[
    P \colon H^0(T^2; \FF_2) \otimes_{\FF_2} H^2(T^2; \FF_2) \to \FF_2
  \]
  where
  \[
    P(1 \otimes z) = \angles{1z, \delta_z}
                   = \angles{z, \delta_z}
                   = \delta_z(z) = 1.
  \]
\end{example}

\begin{example}
  What is \(H^*(\RP^2; \FF_2)\)?
  Note that \(\RP^2\) is a 2D compact manifold.
  We can compute
  \begin{gather*}
    H^0(\RP^2; \FF_2) \iso \FF_2 \\
    H^1(\RP^2; \FF_2) \iso \FF_2
  \end{gather*}
  because \(\RP^2\) has one path component.
  For \(H^1\), we have to manually calculate it.
  However, for \(H^2\), we can use Poincar\'e duality, which gives
  \[
    H^2(\RP^2; \FF_2) \iso H^0(\RP^2; \FF_2) \iso \FF_2.
  \]
  Therefore, we can write
  \[
    H^*(\RP^2; \FF_2) \iso \underbracket{\FF_2\fgen{1}}_{\deg 0} \oplus
                           \underbracket{\FF_2\fgen{a}}_{\deg 1} \oplus
                           \underbracket{\FF_2\fgen{b}}_{\deg 2}.
  \]
  We know that \(b^2 = ab = 0\) for degree reasons, so the only question is
  if \(a^2 = b\) or \(a^2 = 0\).

  There is a perfect pairing
  \[
    P \colon H^1(\RP^2; \FF_2) \otimes_{\FF_2} H^1(\RP^2; \FF_2) \to \FF_2.
  \]
  This map must be nontrivial in order for its adjoint to be an isomorphism.
  Therefore, \(P(a \otimes a) = \angles{a^2, [\RP^2]} \neq 0\), so
  \(a^2 \neq 0\), and furthermore \(a^2 = b\).

  Therefore,
  \[
    H^*(\RP^2; \FF_2) \iso \FF_2[a, b]/(a^2 - b, ab, b^2),
  \]
  where \(\size a = 1\) and \(\size b = 2\).
\end{example}

\begin{remark}
  Note that
  \(H^*(S^2 \vee S^1; \FF_2) \iso \FF_2[a, b]/(a^2, ab, b^2)\)
  where \(\size a = 1\) and \(\size b = 2\).
  So the cohomology groups are the same as \(\RP^2\),
  but the ring structure is different.

  In particular, \(S^2 \vee S^1\) is not a manifold,
  so we cannot apply Poincar\'e duality.
\end{remark}






\end{document}
