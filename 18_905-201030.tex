\documentclass{standalone}
\usepackage{chez}

\begin{document}
\chapter{October 30, 2020}

\begin{theorem}[Eilenberg-Zilber]
  If \(X\) and \(Y\) are topological spaces,
  then the Alexander-Whitney map
  \[
    A \colon S_*(X \times Y) \to S_*(X) \otimes_\ZZ S_*(Y)
  \]
  is a quasi-isomorphism.
\end{theorem}

Recall that \(A\) is a chain map that sends an \(n\)-simplex
\(\sigma \colon \Delta^n \to X \times Y\) to
\[
  A(\sigma) = \sum_{p + q = n} \alpha {\restriction}_{\Delta^p} \otimes
                               \beta  {\restriction}_{\Delta^q}
\]
where \(\alpha {\restriction}_{\Delta^p}\) is the composite
\[
  \Delta^p \xrightarrow{\text{first coordinates}}
  \Delta^n \overset\sigma\to
  X \times Y \overset{p_x}\to
  X
\]
and \(\beta {\restriction}_{\Delta^q}\) is the composite
\[
  \Delta^q \xrightarrow{\text{last coordinates}}
  \Delta^n \overset\sigma\to
  X \times Y \overset{p_y}\to
  Y.
\]

The proof of this theorem is a technical elaboration of
naturality and the fundamental theorem of homological algebra.

\begin{adhoctheorem}{Construction}<construction:M-free-functors>
  Suppose \(\mathcal C\) is a category and
          \(\mathbb M\) is a set of objects of \(\mathcal C\)
            (called the set of \vocab{models}).
  For any \(\set{m_1, \dots, m_r} \subseteq \mathbb M\),
  we can consider the functor
  \begin{align*}
    F \colon& \mathcal C \to \cAb \\[-1ex]
      & c \mapsto \ZZ\fgen[\Big]{\coprod_{m \in \mathbb M} \Hom(m, c)}.
  \end{align*}
  Such a functor is said to be \vocab{\(\mathbb M\)-free}.
\end{adhoctheorem}

\begin{example}
  Let \(\mathcal C = \cTop \times \cTop\).
  Consider \(\mathbb M = \set{\Delta^n \times \Delta^n}\).
  This gives the functor
  \begin{align*}
    F \colon \cTop \times \cAb &\to \cAb \\[-1ex]
      (X, Y) &\mapsto \ZZ\fgen*{
        \Hom_{\cTop \times \cTop}(\Delta^n \times \Delta^n, X \times Y)
      } \\[-1ex]
      &\mapsto \ZZ\fgen*{
        \Hom_{\cTop}(\Delta^n, X) \times \Hom_{\cTop}(\Delta^n, Y)
      } \\[-1ex]
      &\mapsto \ZZ\fgen*{\Hom_{\cTop}(\Delta^n, X \times Y)} \\[-1ex]
      &\mapsto S_n(X \times Y).
  \end{align*}
\end{example}

\begin{example}
  Let \(\mathcal C = \cTop \times \cTop\) again, and consider
  \(
    \set{
      (\Delta^0, \Delta^n),
      (\Delta^1, \Delta^{n-1}), \dots,
      (\Delta^n, \Delta^0)
    }
  \).
  This gives the functor
  \begin{align*}
    F(X, Y) &= \ZZ\fgen[\Big]{\coprod_{p + q = n}
      \Hom_{\cTop \times \cTop}((\Delta^p, \Delta^q), (X, Y))} \\
    &= \ZZ\fgen[\Big]{\coprod_{p + q = n}
      \Hom_{\cTop}(\Delta^p, X) \times \Hom_{\cTop}(\Delta^q, Y)} \\
    &\iso \bigoplus_{p + q = n} \ZZ\fgen[\Big]{
      \Hom_{\cTop}(\Delta^p, X) \times \Hom_{\cTop}(\Delta^q, Y)} \\
    &\iso \bigoplus_{p + q = n} \ZZ\fgen[\Big]{S_p(X) \otimes_\ZZ S_q(Y)} \\
    &\iso (S_*(X) \otimes_\ZZ S_*(Y))_n.
  \end{align*}
\end{example}

\begin{definition}
  Let \(\mathcal C\) be a category and
      \(\mathbb M\) be a set of objects in \(\mathcal C\).
  Suppose \(F \colon \mathcal C \to \cAb\) be any functor.
  An \vocab{\(\mathbb M\)-free resolution} of \(F\) is
    a functor \(F_*\colon \mathcal C \to \cchAb\) with
    a natural transformation \(\eps_F \colon H_0(F_*) \to F\) satisfying:
  \begin{enumerate}[nosep]
    \item Each \(F_n\) is \(\mathbb M\)-free.
      In particular \(F_*\) composed with taking the \(n\)th group is
      an \(\mathbb M\)-free functor \(F_n \colon \mathcal C \to \cAb\).
    \item When evaluated on an object \(m \in \mathbb M\),
      \(F_*(m)\) is a free resolution of \(F(m)\) in the sense that
      \[
        H_i(F_*(m)) = \begin{cases}
          0 & i \neq 0 \\[-1ex]
          F(m) & i = 0
        \end{cases}
      \]
      and \(\eps_F \colon H_0(F_*(m)) \to F(m)\) is an isomorphism.
  \end{enumerate}
\end{definition}

\begin{theorem}
  Let \(\Theta \colon F \to G\) be a natural transformation of
  functors \(F, G \colon \mathcal C \to \cAb\).
  If \(F_*\) and \(G_*\) are \(\mathbb M\)-free resolutions of \(F\) and \(G\),
  there is a natural transformation \(\Theta_* \colon F_* \to G_*\)
  lifting \(\Theta\) that is unique up to natural chain homotopy.
\end{theorem}

\begin{corollary}
  The Eilenberg-Zilber theorem is true and
  the Alexander-Whitney map \(A\) is unique up to natural chain homotopy.
\end{corollary}
\begin{proof}
  Let \(\mathcal C = \cTop \times \cTop\) and
  \begin{align*}
    F(X, Y) &= H_0(X \times Y) \\
    G(X, Y) &= H_0(S_*(X) \otimes_\ZZ S_*(Y)) \\
    F_*(X, Y) &= S_*(X \times Y) \\
    G_*(X, Y) &= S_*(X) \otimes_\ZZ S_*(Y).
  \end{align*}
  Note that when \(X\) and \(Y\) are both simplices,
  both \(F_*\) and \(G_*\) record chain complexes
  with homology only in degree zero.
  On the model objects, \(f_*\) and \(G_*\) are
  free resolutions of \(F\) and \(G\).
  In general, \(F_*\) and \(G_*\) are \(\mathbb M\)-free resolutions.

  It follows that if we have a map
  \(H_0(X \times Y) \to H_0(S_*(X) \otimes_\ZZ S_*(Y))\),
  there is a unique up to natural chain homotopy lift to
  the Alexander-Whitney map
  \[
    A \colon S_*(X \times Y) \to S_*(X) \otimes_\ZZ S_*(Y).
  \]
  This means that the Alexander-Whitney map exists,
  and moreover that it is unique.
  In particular, if we want to produce an Alexander-Whitney map,
  it suffices to produce the map between the map between their zeroth
  homology and then lift the map.
  However, this is not too hard because zeroth homologies
  are just connected components.

  To prove uniqueness, consider a natural isomorphism
  \[
    H_0(X \times Y) \to H_0(S_*(X) \otimes_\ZZ S_*(Y))
  \]
  and a natural inverse
  \[
    H_0(S_*(X) \otimes_\ZZ S_*(Y)) \to H_0(X \times Y).
  \]
  Composing these maps gives the identity on either
  \(H_0(X \times Y)\) or \(H_0(S_*(X) \otimes_\ZZ S_*(Y))\),
  and it follows by the fundamental theorem of homological algebra,
  \(S_*(X \times Y)\) and \(S_*(X) \otimes_\ZZ S_*(Y)\)
  are chain homotopy equivalent.
\end{proof}

\section{The diagonal map}
Suppose \(X \in \cTop\).
The diagonal map is the map \(\Delta \colon X \to X \times X\)
mapping \(x \mapsto (x, x)\).

If \(X\) is a space which has torsion free homology groups,
i.e.\ the homology groups are free abelian groups.
Then from K\"unneth,
\[
  H_n(X \times X) = \bigoplus_{p + q = n} H_p(X) \otimes_\ZZ H_q(X),
\]
where all of the \(\Tor\) groups are \(0\) because the homologies are free.
Let \(H_*(X) \coloneqq \bigoplus_n H_n(X)\).

\begin{example}
  \(H_*(T^2) = H_0(T^2) \oplus H_1(T^2) \oplus H_2(T^2)
             = \ZZ \oplus (\ZZ \oplus \ZZ) \oplus \ZZ\).
\end{example}

Then we have
\begin{align*}
  H_*(X) \otimes_\ZZ H_*(X)
    &\iso \parens[\Big]{\bigoplus_p H_p(X)} \otimes_\ZZ
          \parens[\Big]{\bigoplus_q H_q(X)} \\
    &\iso \bigoplus_{p, q} H_p(X) \otimes_\ZZ H_q(X) \\
    &\iso \bigoplus_n \bigoplus_{p + q = n} H_p(X) \otimes_\ZZ H_q(X) \\
    &\iso \bigoplus_n H_n(X \times X) \\
    &\iso H_*(X \times X).
\end{align*}
Therefore, \(\Delta \colon X \to X \times X\) induces a map
\[
  H_*(X) \to H_*(X \times X) \iso H_*(X) \otimes_\ZZ H_*(X).
\]

\begin{question}
  In algebra, what does it mean to have an abelian group \(A\)
  and a map \(A \to A \otimes_\ZZ A\)?
  \tcblower
  It is called a comultiplication, which is the dual of multiplication.
\end{question}

\begin{definition}
  If \(A\) is an abelian group, a \vocab{multiplication}
  is a map of abelian groups
  \[
    A \otimes_\ZZ A \to A.
  \]
\end{definition}

Note that a map \(m \colon A \otimes_\ZZ A \to A\) is a rule
that takes a pair \(a \otimes a'\) to an element \(a'' = m(a \otimes a')\).
Note that we have
\begin{gather*}
  m((a_1 + a_2) \otimes a') = m(a_1 \otimes a') + m(a_2 \otimes a') \\
  m(a \otimes (a'_1 + a'_2)) = m(a \otimes a'_1) + m(a \otimes a'_2).
\end{gather*}

\begin{example}
  A \vocab{ring} is an abelian group \(A\) with a multiplication satisfying
  some properties.
\end{example}

The fact that \(\Delta\) induces a comultiplication instead of a multiplication
is not connected to more familiar mathematical structures.






\end{document}
