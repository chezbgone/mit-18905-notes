\documentclass{standalone}
\usepackage{chez}

\begin{document}
\chapter{September 16, 2020}

\begin{question}
  Given two maps \(f, g \colon X \to Y\) in \(\cTop\), when are
  \(S_*(f)\) and \(S_*(f)\) chain homotopic?
  When they are, we learn that \(H_n(f) = H_n(g)\).
\end{question}

\begin{definition}
  A \vocab{homotopy} \(h\) between \(f, g \colon X \to Y\) is a continuous map
  \(h \colon X \times [0, 1] \to Y\) such that \(h(x, 0) = f(x)\) and
  \(h(x, 1) = g(x)\).

  We say that \(f\) and \(g\) are \vocab{homotopic} if there exists a homotopy
  from \(f\) to \(g\).
\end{definition}

Often we think of the second variable as time, and we can think of \(h\) as
a family of maps that interpolates from \(f\) to \(g\), such that at \(t = 0\),
we have the map \(f\) and at \(t = 1\), we have \(g\).

\begin{theorem}
  If \(f\) and \(g\) are homotopic, then \(S_*(f)\) and \(S_*(g)\) are
  chain homotopic. Hence, \(H_n(f) = H_n(g)\) as maps in \(\cAb\).
\end{theorem}
The proof of this is on PSet 2.

Suppose \(h_{12} \colon X \times [0, 1] \to Y\) is a homotopy
from \(f_1\) to \(f_2\), and \(h_{12} \colon X \times [0, 1] \to Y\)
is a homotopy from \(f_2\) to \(f_3\).
Then we can define \(h_{13} \colon X \times [0, 1] \to Y\) where
\[
  h_{13}(x, t) = \begin{cases}
    h_{12}(x, 2t) & 0 \leq t \leq 1/2 \\[-1ex]
    h_{23}(x, 2t-1) & 1/2 \leq t \leq 1.
  \end{cases}
\]
This \(h_{13}\) is a homotopy from \(f_1\) to \(f_3\), implying that homotopy
equivalence is transitive. Indeed, if we define \(f \simeq g\)
if \(f\) and \(g\) are homotopic, then \(\simeq\) is an equivalence relation
on the set \(\Hom_{\cTop}(X, Y)\).

Furthermore, there is a category \(\cHoTop\) called
the homotopy category where
\begin{gather*}
  \Obj \cHoTop = \Obj \cTop \\
  \Hom_{\cHoTop}(X, Y) = \Hom_{\cTop}(X, Y) / {\simeq}.
\end{gather*}
In particular, the maps in the homotopy category \(\cHoTop\) are
homotopy classes of maps in \(\cTop\).

Note that there is a canonical functor \(\cTop \to \cHoTop\), namely
sending a topological space to itself, and
sending a continuous map to the equivalence class containing that continuous
map under the homotopy relation.

A consequence of the theorem is that there is a diagram of functors
\[
  \begin{tikzcd}[column sep=tiny]
    \cTop \ar[rr, "H_n"]\ar[rd] && \cAb \\
    & \cHoTop \ar[ru] &
  \end{tikzcd}
\]
that commutes. In particular, we claim that there exists a unique functor
\(\cHoTop \to \cAb\).

\begin{adhoctheorem*}{Aside on fundamental groups}
  The fundamental group is similar in that we can form the diagram
  \[
    \begin{tikzcd}[column sep=tiny]
      \cTop_* \ar[rr, "\pi_1"] \ar[rd] && \cGrp \\
      & \cHoTop \ar[ru] &
    \end{tikzcd}
  \]
  where \(\cTop_*\) is the category of topological spaces with a base point.
\end{adhoctheorem*}

\begin{definition}
  A continuous map \(f \colon X \to Y\) of topological spaces is called a
  \vocab{homotopy equivalence} if it maps to an isomorphism under the functor
  \(\cTop \to \cHoTop\), i.e. there exists a map \(g \colon Y \to X\) such that
  \(f \circ g \simeq 1_Y\) and \(g \circ f \simeq 1_X\).

  Two spaces \(X, Y\) are \vocab{homotopy equivalent} if there exists
  a homotopy equivalence \(f \colon X \to Y\).
\end{definition}

More explicitly, \(f\) is a homotopy equivalence if and only if there exists
an inverse \(g \colon Y \to X\) such that \(f \circ g \simeq 1_Y\) and
\(g \circ f \simeq 1_X\). Note that we only require these compositions to
be homotopic to the identity, as opposed to equal to the identity, which
describes homeomorphism. An important distinction to make is that maps
can be homotopic, but spaces are homotopy equivalent.

\begin{example}
  The maps \(f \colon \RR \to *\) and \(g \colon * \to \RR\) where
  \(f \colon x \mapsto *\) and \(g \colon * \mapsto 0\) compose to
  \(f \circ g \colon * \mapsto *\) and \(g \circ f \colon x \mapsto 0\),
  which are homotopic to their respective identities.

  Therefore, \(\RR\) and \(*\) are homotopy equivalent, but not homeomorphic.
\end{example}

\begin{fact}
  The thing to remember here is that homology cannot tell the difference
  between homotopy equivalent spaces or homotopic maps. In particular,
  homology is an invariant on a weaker notion of equivalence, namely
  homotopy equivalence.
\end{fact}

It's useful to get a geometric feeling for when two non-homeomorphic spaces
are homotopy equivalent. The following way is a simple way to check for
homotopy equivalence a lot of the time.

\begin{definition}
  An inclusion \(A \hookrightarrow X\) is a \vocab{deformation retraction}
  provided that there is a map \(h \colon X \times [0, 1] \to X\) such that
  \begin{enumerate}[nosep]
    \item \(h(x, 0) = x\) for all \(x \in X\),
    \item \(h(x, 1) \in A\) for all \(x \in X\),
    \item \(h(a, t) = a\) for all \(a \in A\) and \(t \in [0, 1]\).
  \end{enumerate}
\end{definition}
Intuitively, we can again think of the second coordinate \(t\) as time,
where at \(t = 0\), we have the space \(X\), and as time goes on, we retract
everything into \(A\), while not moving anything in \(A\) throughout
the process.

\begin{example}
  \(S^{n-1} \subset \RR^n - \set{0}\) is a deformation retraction.
  In particular, we can take each point \(x\) in \(\RR^n - \set{0}\)
  and move it to the point on \(S^{n - 1}\) that intersects the ray from the
  origin passing through \(x\).
\end{example}

One way to prove that two spaces \(X\) and \(Y\) are homotopy equivalent is
to exhibit a common deformation retraction:
\[
  \begin{tikzcd}
    X & A \ar[l, hook'] \ar[r, hook] & Y
  \end{tikzcd}
\]

We know that if \(A \simeq B\) are homotopy equivalent spaces, then they
have the same homology.
Last class, we proved that star-shaped regions are homotopy equivalent to
a point. Our next goal is to answer the following question:
\begin{question}
  If \(A \subseteq X\) is the inclusion of a subspace, how does \(H_n(A)\)
  compare to \(H_n(X)\)?
\end{question}

\begin{claim}
  If \(A \subseteq X\) is the inclusion of a subspace, \(S_*(A) \to S_*(X)\)
  is a subcomplex.
\end{claim}
\begin{definition}
  A \vocab{subcomplex} \(C_* \subseteq D_*\) is a diagram
  \[
    \begin{tikzcd}
      \cdots \arrow[r] &
        C_2 \arrow[r] \arrow[d, symbol=\subseteq] &
        C_1 \arrow[r] \arrow[d, symbol=\subseteq] &
        C_0 \arrow[r] \arrow[d, symbol=\subseteq] &
        C_{-1} \arrow[r] \arrow[d, symbol=\subseteq] &
        C_{-2} \arrow[r] \arrow[d, symbol=\subseteq] &
        \cdots \\
      \cdots \arrow[r] &
        D_2 \arrow[r] &
        D_1 \arrow[r] &
        D_0 \arrow[r] &
        D_{-1} \arrow[r] &
        D_{-2} \arrow[r] &
        \cdots
    \end{tikzcd}
  \]
  where the \(\subseteq\)s are inclusion maps and all squares commute.
\end{definition}

When we have an inclusion, then we can take the quotient.

\begin{adhoctheorem}{Construction}
  Suppose \(C_* \subseteq D_*\) is a subcomplex. The quotient complex
  \(D_*/C_*\) has groups \((D_*/C_*)_n = D_n/C_n\), where the differential
  \(\partial_n \colon D_n/C_n \to D_{n-1}/C_{n-1}\) is well defined because
  if two classes in \(D_n\) differ by a class in \(C_n\), then their boundaries
  differ by a class in \(C_{n-1}\). This is the same as the condition that all
  the squares commute.
\end{adhoctheorem}

\end{document}
