\documentclass{standalone}
\usepackage{chez}

\begin{document}
\chapter{September 11, 2020}
Today we will finish putting semisimplicial sets into
a category theory viewpoint.

Suppose \(\mathcal C\) is a category. Then a new category we can consider is
the \vocab{opposite category} \(\mathcal C^{\mathrm{op}}\) with
\begin{itemize}[nosep]
  \item \(\Obj(\mathcal C^\op) = \Obj(\mathcal C)\)
  \item For \(X, Y \in \Obj(\mathcal C^\op)\),
  \(\Hom_{\mathcal C^\op}(X, Y) = \Hom_{\mathcal C}(Y, X)\).
\end{itemize}
In particular, if \(f \in \Hom_{\mathcal C}(Y, X)\), we use \(f^\op\)
to represent the corresponding object of \(\Hom_{\mathcal C^\op}(X, Y)\).
The composition law is
\[
  (f \circ g)^\op = g^\op \circ f^\op.
\]

A good way to think about the opposite category is through functors.
As a category itself, \(\mathcal C^\op\) is essentially the same as
\(\mathcal C\), but a functor \(F \colon \mathcal C^\op \to \mathcal D\)
can be thought of as a function that still takes objects in \(\mathcal C\)
to objects in \(\mathcal D\), but takes maps \(c \to c'\) in \(\mathcal C\)
to maps \(F(c') \to F(c)\) in \(\mathcal D\).


\begin{example}
  Recall the category \(\cat{Vect}_\RR\) of real vector spaces.
  Every vector space \(V\) has a dual
  \(V^* = \underline{\Hom}_{\cat{Vect}_\RR}(V, \RR)\),
  where the underline indicates that the \(\Hom\) is a vector space itself.

  Note that if \(f \colon W \to V\) is a linear map, then the corresponding
  map \(f^* \colon V^* \to W^*\) in the opposite category maps
  \[
    f^* \colon
    \underset{\mathclap{\in V^*}}{(g \colon V \to \RR)}
    \mapsto
    \underset{\mathclap{\in W^*}}{(g \circ f \colon W \to V \to \RR)}.
  \]
  
  In particular, we have the functor
  \((\,)^* \colon \cat{Vect}_\RR \to \cat{Vect}_\RR^\op\).
\end{example}

From last time, recall that for categories \(\mathcal C, \mathcal D\)
where \(\mathcal C\) has a set of objects, \(\Fun(\mathcal C, \mathcal D)\)
is a category where the objects are the set of functors
\(\mathcal C \to \mathcal D\), and morphisms are
natural transformations of functors. In particular, we can
compose natural transformations.

\section{Semisimplicial sets}
\begin{definition}
  Let \(\Deltainj\) be the category with objects
  \[
    \Obj \Deltainj = \set{[0], [1], [2], \dots},
  \]
  and morphisms
  \[
    \Hom_{\Deltainj}([a], [b]) = \set*{
      \parbox{7.8cm}{
        injective functions
        \(f \colon \set{0, 1, \dots, a} \to \set{0, 1, \dots, b}\)
        that preserve order, i.e. \(f(x) < f(y)\) whenever \(x < y\)
      }
    }.
  \]
\end{definition}

For instance, there are three maps in \(\Hom_{\Delta}([1], [2])\), which are
injective order preserving functions
\[
  \set{0, 1} \to \set{0, 1, 2}.
\]
They are determined by their image \((f(0), f(1))\),
which could be \((0, 1)\), \((0, 2)\), or \((1, 2)\).

\begin{claim}
  A semisimplicial set is a functor \(\Deltainj^\op \to \cat{Set}\),
  i.e.\ an element of \(\Fun(\Deltainj^\op, \cat{Set})\).
\end{claim}

Let's unpack this. A semisimplicial is a sequence of sets
\[
  X_0, X_1, X_2, \dots
\]
with maps:
\begin{align*}
  d_0, d_1 &\colon X_1 \to X_0 \\
  d_0, d_1, d_2 &\colon X_2 \to X_1 \\
  \MTFlushSpaceAbove
  &\vdotswithin{\colon}
\end{align*}
such that \(d_i d_j = d_{j-1}d_i\) when \(i < j\).

We are looking for a functor \(F \colon \Deltainj^\op \to \cat{Set}\).
We have
\[
  \Obj \Deltainj^\op = \Obj \Deltainj = \set{[0], [1], \dots}.
\]
The most natural way to do this is to map \(F([n])\) to \(X_n\) in the
semisimplicial set. To get the \(d_i\) maps, consider that
\[
  \Hom_{\Deltainj^\op}([2], [1]) = \Hom_{\Deltainj}([1], [2])
     = \set*{\parbox{4.5cm}{
      injective order preserving maps \(\set{0, 1} \to \set{0, 1, 2}\)
     }}.
\]
Recall that there are three of these.
Let \(f_0 \colon \set{0, 1} \to \set{0, 1, 2}\) be the map with image
\(\set{1, 2}\), i.e.\ the one that does not contain \(0\).
In general, let \(f_k \colon \set{0, 1} \to \set{0, 1, 2}\) be the map with
image \(\set{0, 1, 2} \setminus \set{k}\).
Then, we can map \(F(f_k^\op)\) to \(d_k\) in the semisimplicial set.

It turns out that the combinatorics of these injective morphisms translates
directly into the simplicial identity \(d_i d_j = d_{j-1} d_i\).

\begin{remark}
  \(\Fun(\Deltainj^\op, \cat{Set})\) forms a category of semisimplicial sets
  where the morphisms are natural transformations.
\end{remark}

\begin{theorem}
  There are functors
  \begin{align*}
    \Sing &\colon \cat{Top} \to \Fun(\Deltainj^\op, \cat{Set}) \\
    S_n, Z_n, B_n, H_n &\colon \Fun(\Deltainj^\op, \cat{Set}) \to \cat{Ab}.
  \end{align*}
\end{theorem}

\begin{corollary}
  There are functors
  \[
    S_n, Z_n, B_n, H_n \colon \cat{Top} \to \cat{Ab}.
  \]
\end{corollary}
Note that this corollary was deduced from the fact that \(\cat{Cat}\)
is a category.

\begin{definition}
  Let \(\cat{Fil}\) (standing for filtered) denote the category with one object
  for each nonnegative integer, and where \(\Hom(a, b)\) is empty if \(a < b\),
  and \(\Hom(a, b)\) has a unique element if \(a \geq b\).
\end{definition}
Note that the data of a functor \(\cat{Fil} \to \cat{Ab}\) is a sequence of
abelian groups with maps between them
\[
  \begin{tikzcd}
    A_0 &
    A_1 \ar[l, "\partial_1", swap] &
    A_2 \ar[l, "\partial_2", swap] &
    A_3 \ar[l, "\partial_3", swap] &
    \cdots \ar[l, "\partial_4", swap]
  \end{tikzcd}
\]

\begin{definition}
  A \vocab{nonnegative chain complex} of abelian groups is
  a functor \(\cat{Fil} \to \cat{Ab}\) with the property that
  \(\partial_{i - 1} \circ \partial_i = 0\) for all \(i \geq 2\).
\end{definition}

Note that there is a category \(\cat{chAb}_{\geq 0}\) of
nonnegative chain complexes where the morphisms are
natural transformations of functors \(\cat{Fil} \to \cat{Ab}\).
More explicitly, a map of nonnegative chain complexes is a diagram
\[
  \begin{tikzcd}
    A_0 \ar[d] &
      A_1 \ar[l, "\partial_1", swap] \ar[d] &
      A_2 \ar[l, "\partial_2", swap] \ar[d] &
      A_3 \ar[l, "\partial_3", swap] \ar[d] &
      \cdots \ar[l, "\partial_4", swap] \\
    B_0 &
      B_1 \ar[l, "\partial_1"] &
      B_2 \ar[l, "\partial_2"] &
      B_3 \ar[l, "\partial_3"] &
      \cdots \ar[l, "\partial_4"]
  \end{tikzcd}
\]
where all the squares commute.

\begin{claim}
  There is a functor
  \begin{align*}
    S_* \colon& \Fun(\Deltainj^\op, \cat{Set}) \to \cat{chAb}_{\geq 0} \\
      & X \mapsto \parens*{
        \begin{tikzcd}[ampersand replacement=\&]
          S_0(X) \&
          S_1(X) \ar[l, "\partial_1", swap] \&
          S_2(X) \ar[l, "\partial_2", swap] \&
          S_3(X) \ar[l, "\partial_3", swap] \&
          \cdots \ar[l, "\partial_4", swap]
        \end{tikzcd}
      }
  \end{align*}
  mapping a semisimplicial set to the chain complex of singular \(n\)-chains,
  which are the free abelian groups generated by the \(n\)-simplices \(X_n\).
\end{claim}

\begin{theorem}
  There are functors \(Z_n, B_n, H_n \colon \cat{chAb}_{\geq 0} \to \cat{Ab}\).
\end{theorem}

In summary, the \(H_n \colon \cat{Top} \to \cat{Ab}\) is a composite of
\[
  \begin{tikzcd}
    \cat{Top} \ar[r, "\Sing"] &
    \Fun(\Deltainj^\op, \cat{Set}) \ar[r, "S_*"] &
    \cat{chAb}_{\geq 0}
    \ar[rr, "H_n = \ker(\partial_n)/\img(\partial_{n+1}) = Z_n/B_n"] &
    {\hspace{1.8cm}} &
    \cat{Ab}
  \end{tikzcd}
\]



\end{document}
